\documentclass[12pt]{article}
\usepackage{geometry,fancyhdr,xr,hyperref,ifpdf,amsmath,rcs,shortcuts,amsfonts}
\usepackage{lastpage,longtable,color,paunits,amsmath,smallsec}
%\usepackage{showlabels}
\geometry{letterpaper,top=50pt,hmargin={20mm,20mm},headheight=15pt} 


\pagestyle{fancy} 

\RCS $Revision: 1.1 $

\RCS $Date: 2007/11/23 17:26:55 $
\fancypagestyle{first}{
\lhead{}
\chead{Pressure perturbations}
\rhead{page~\thepage/\pageref{LastPage}}
\lfoot{} 
\cfoot{} 
\rfoot{}
}

\ifpdf
    \usepackage[pdftex]{graphicx} 
    \usepackage{hyperref}
    \pdfcompresslevel=0
    \DeclareGraphicsExtensions{.pdf,.jpg,.mps,.png}
\else
    \usepackage{hyperref}
    \usepackage[dvips]{graphicx}
    \DeclareGraphicsRule{.eps.gz}{eps}{.eps.bb}{`gzip -d #1}
    \DeclareGraphicsExtensions{.eps,.eps.gz}
\fi

\newcommand{\vect}[1]{\mathbf{{#1}}}

\begin{document}

\pagestyle{first}

\begin{center}
Atsc500: The pressure perturbation\\
\end{center}


\textit{below I'm following the argument in Garratt, J. R., The atmospheric boundary layer, p. 36}

In section 4.3 (p.~123) Stull derives the equation for the time evolution of 
the perturbation kinetic energy/kg  which is
 $\frac{1 }{2} \left [ {u_1^\prime}^2 + {u_2^\prime}^2 + {u_3^\prime}^2 \right ]$.  This produces 
pressure perturbation terms of the form:

\begin{gather}
  -2 \overline{  \left (  \frac{ {u_i^\prime}  }{ \rho}   \right )  \frac{\partial p^\prime  }{\partial x_i} }
\end{gather}
which Roland rewrites using the chain rule as:

\begin{gather}
    -2 \overline{  \left (  \frac{ {u_i^\prime}  }{ \rho}   \right )  \frac{\partial p^\prime  }{\partial x_i} }
=  \underbrace{  - \frac{2 }{\rho} \frac{\partial \overline{  (u_i^\prime p^\prime )}  }{\partial x_i }  }_I
+ \underbrace{
2 \overline{ \left (  \frac{p^\prime  }{\rho}  \right ) \left (  \frac{\partial   u_i^\prime }  {\partial x_i }  \right ) }}_{II}
\end{gather}

He notes that term (II) sums to zero, because the incompressible Bousinesq continuity
equation requires that $\nabla \cdot \vect{u} =0$.  Term (II) is still
important though, because it ``switches'' the kinetic energy between
components, producing what is sometimes called ``return to isotropy''.

First note Stull's derivation (eq.~4.3.1b, p. 122) where he shows that the dissipation rate $\epsilon$ 
can be written as: 

\begin{gather}
 2 \nu u_i\prime \frac{\partial^2 \overline{ {u_i^\prime}^2 } }{\partial x_j \partial x_j}     = 
2 \nu \overline{ \vect{u} \cdot \nabla^2 \vect{u}    }  \approx
- 2 \nu \overline{ \nabla \vect{u} \cdot \nabla \vect{u}    }
 = - 2 \nu \overline{ \left (  \frac{\partial u^\prime_i }{\partial x_j}  \right )^2 } = - 2 \epsilon < 0
\end{gather}

Now consider a boundary layer in which there is  $u(z)$ component, with no variation in $x$ or $y$.  
Since $\epsilon$ only
becomes large on the smallest scales, and since turbulence on the smallest scale is isotropic,
we will divide $2 \epsilon$ evenly between the three velocity components.  Neglecting the 
Coriolis force, the three components of 4.3.1a can be written:

\begin{subequations}
\label{eq:perturb}
\begin{align}
  \frac{\partial  \overline{ u^{^\prime 2}}}{\partial t} = & 
- \overline{ u^\prime w^\prime  } \frac{\partial  \overline{u } }{\partial  z} &
+ \frac{ 1}{\rho} \left (    \overline{ p^\prime \frac{ \partial u^\prime }{\partial x}  }  \right ) & - \frac{2 \epsilon }{3} \label{eq:up}\\
  \frac{\partial  \overline{ v^{^\prime 2}}}{\partial t} = & \text{ }
  &
+ \frac{ 1}{\rho} \left (   \overline{ p^\prime \frac{ \partial v^\prime }{\partial y}  }  \right ) & - \frac{2 \epsilon }{3}\label{eq:vp}\\
  \frac{\partial  \overline{ w^{^\prime 2}}}{\partial t} = & \text{ }
  &  
+ \frac{ 1}{\rho} \left (   \overline{ p^\prime \frac{ \partial w^\prime }{\partial z}  }  \right )  & - \frac{2 \epsilon }{3}\label{eq:wp}
\end{align}
\end{subequations}

What does this tell us about the $p^\prime$ terms?  We already know that they sum to zero.  We also know
that at the smallest scales, where $\epsilon$ is large (and positive) and turbulence is
isotropic,
$\overline{ u^{^\prime 2}}, \overline{ v^{^\prime 2}}$ and 
$\overline{ w^{^\prime 2}}$ are all about the
same size, so the time derivatives for $\overline{ u^{^\prime 2}}$ and  
$\overline{ v^{^\prime 2}}$ can't be
negative, or horizontal turbulence will fall to zero.   That means
that the pressure perturbation terms in (\ref{eq:vp}) and (\ref{eq:wp})
have to be positive to achieve steady state.
And again because  $\epsilon$ is positive, that means that the only source
of TKE in this system is $\overline{ u^\prime w^\prime  } \frac{\partial  \overline{u } }{\partial  z}$, i.e. all of the turbulent energy is entering through
(\ref{eq:up}).
So if we 
 write the individual pressure perturbation terms as $P=\frac{2 \epsilon }{3}$ 
then $P$ has to
be positive in equations (\ref{eq:wp}) and (\ref{eq:wp}) and in
steady state 
$  \frac{\partial  \overline{ v^{^\prime 2}}}{\partial t}$ =
$  \frac{\partial  \overline{ w^{^\prime 2}}}{\partial t}$=0, 
we know that $P = \frac{2 \epsilon }{3} > 0$. Again, since 
the sum of the total pressure perturbation across all components
has to be 0, we know that $\frac{ 1}{\rho} \left (    \overline{ p^\prime \frac{ \partial u^\prime }{\partial x}  } \right )$ needs to be $-2P$ to make this
happen.  So  rewrite
(\ref{eq:perturb}) as:

\begin{subequations}
\label{eq:perturb2}
\begin{align}
  \frac{\partial  \overline{ u^{^\prime 2}}}{\partial t} = & 
- \overline{ u^\prime w^\prime  } \frac{\partial  \overline{u } }{\partial  z} &
-  2 P & - \frac{2 \epsilon }{3} \\
  \frac{\partial  \overline{ v^{^\prime 2}}}{\partial t} = & \text{ }
  &
+ P & - \frac{2 \epsilon }{3}\\
  \frac{\partial  \overline{ w^{^\prime 2}}}{\partial t} = & \text{ }
  & + P  & - \frac{2 \epsilon }{3}
\end{align}
\end{subequations}

Equation (\ref{eq:perturb2}) is saying that, as $u^\prime$ grows, it
develops a large negative correlation between the horizontal shear
of the turbulence and the pressure perturbation:
$2P = - \frac{ 1}{\rho} \left (    \overline{ p^\prime \frac{ \partial u^\prime }{\partial x}  }  \right )$.  This removes TKE from the horizontal, and
distributes it into the meridional and vertical components, 
keeping the turbulence isotropic.





\end{document}


%%% Local Variables:
%%% mode: latex
%%% TeX-master: t
%%% End:
