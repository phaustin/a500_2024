\documentclass[12pt]{article}
\usepackage{geometry,fancyhdr,xr,hyperref,ifpdf,amsmath,rcs,indentfirst}
\usepackage{lastpage,longtable,Ventry,url,paunits,shortcuts,smallsec,color,tightlist,float}
\geometry{letterpaper,top=50pt,hmargin={20mm,20mm},headheight=15pt}
\usepackage[stable]{footmisc}

\pagestyle{fancy} 

\RCS $Revision: 1.5 $
\RCS $Date: 2002/01/09 03:50:54 $

\fancypagestyle{first}{
\chead{Chapter 1: Eulerian and Lagrangian derivatives}
\lhead{2019/Sep/10}
\rhead{page~\thepage/\pageref{LastPage}}
\lfoot{} 
\cfoot{} 
\rfoot{}
}

\ifpdf
    \usepackage[pdftex]{graphicx} 
    \usepackage{hyperref}
    \pdfcompresslevel=0
    \DeclareGraphicsExtensions{.pdf,.jpg,.mps,.png}
\else
    \usepackage{hyperref}
    \usepackage[dvips]{graphicx}
    \DeclareGraphicsRule{.eps.gz}{eps}{.eps.bb}{`gzip -d #1}
    \DeclareGraphicsExtensions{.eps,.eps.gz}
\fi

\begin{document}
\pagestyle{first}


\section{Total (or material) derivative}
\label{sec:total-derivitive}

Make sure you understand
the important distinction between the total and partial derivative
on page 6 of Stull Chapter 1.   We need them both, because we use the
total derivitive to write \textit{conservation laws} (for mass, momentum,
energy and entropy) in a reference frame moving with the flow, but
models and observations are generally constructed/measured in a fixed 
(eulerian) reference frame with $x,y,z$ constant. Here are two different approaches
to deriving the p.~6 expression.  
For the first derivation it helps to  to recall some basic facts about
\href{http://en.wikipedia.org/wiki/Taylor_series}{Taylor's series}
(actually Gregory's series) and
\href{http://bit.ly/TaylorsTheorem}{Taylor's theorem}

In particular in the neighborhood of the point $t=t_0$, $x=a$, $y=b$

\begin{eqnarray*}
f(x,y)
&\approx& f(a,b,t_0) + (t-t_0)\frac{\partial f }{\partial t}(a,b,t_0) +(x-a)\frac{\partial f }{\partial x}(a,b,t_0) +(y-b)\, \frac{\partial f }{\partial y}(a,b,t_0) \\
&+& \mathrm{higher\ order\ terms}
\end{eqnarray*}
or keeping only terms of first order and rearranging

\begin{eqnarray*}
f(x,y,t) - f(a,b,t_0) = \Delta f
&\approx& \Delta t \frac{\partial f }{\partial t}(a,b,t_0) +  \Delta x \frac{\partial f }{\partial x}(a,b,t_0) + \Delta y\, \frac{\partial f }{\partial y}(a,b,t_0)
\end{eqnarray*}
and dividing through by $\Delta t$:

\begin{eqnarray*}
\frac{\Delta f }{\Delta t} 
&\approx& \frac{\Delta t }{\Delta t} \frac{\partial f }{\partial t}(a,b,t_0) +  \frac{\Delta x }{\Delta t} \frac{\partial f }{\partial x}(a,b,t_0) + \frac{\Delta x }{\Delta t} \frac{\partial f }{\partial y}(a,b,t_0) \\
&=& \frac{\partial f }{\partial t} +  \boldmath{u} \cdot \nabla f
\end{eqnarray*}

But how do we know that $\frac{\Delta f }{\Delta t}$ is the \textit{material} derivative -- i.e. the
derivative in a reference frame moving with the flow?  Take a look 
at two fluid mechanics lectures by
\href{http://star-www.st-and.ac.uk/~mmj}{Moira Jardine} of the University
of St.~Andrews, \href{https://www.dropbox.com/s/e2q2eruf7d0kk4w/fluids_01.pdf?dl=0}{Fluid lecture 1} and
\href{https://www.dropbox.com/s/7d89qa4v2kus0r6/fluids_02.pdf?dl=0}{Fluid lecture 2}.

Her definition of $\frac{ dQ}{dt}$ is:


\begin{equation}
  \label{eq:dqdt}
  \frac{ dQ}{dt} = \frac{ Q(r + \delta r, t + \delta t) - Q(r,t) }{\delta t}
\end{equation}
with the implicit idea that $Q(r + \delta r, t + \delta t)$ is the same piece of fluid
as $Q(r,t) {\delta t}$, observed at time $t + \delta t$ -- that is, the displacement
$\delta r$ is whatever is needed to track the fluid parcel.

If this seems a little ad hoc,
it is possible to make this more
explicit by specifically labeling the fluid volumes by their
initial positions, and being clear about the fact that we
are following the volumes as time progresses.  Here I'm
relying on a derivation 
given in Salmon, 1998, Lectures on Geophysical Fluid Dynamics.

In Salmon's notation, for the
\textit{Eulerian description}, the independent variables are the space
coordinates $\mathbf{x}=(x,y,z)$ and the time $t$.  In the
\textit{Lagrangian description}, the independent coordinates are a
set of particle labels which uniquely identify a small region of the
fluid.  For example the label could be the three coordinates
$\mathbf{a}=(a,b,c)$, where $(a,b,c)$ have the numerical values
of $(x,y,z)$ at time $t=0$.  It is also helpful to give a separate
symbol to the time in the Lagrangian frame, $\tau$, to remind us that
as $\tau$ varies the label $(a,b,c)$ is kept fixed, while as $t$, varies
the position $(x,y,z)$ is kept fixed.  In this notation the material
or total derivative is by definition the time derivitive keeping the
labels constant:

\begin{equation}
  \label{eq:material}
  \frac{\partial Q}{\partial \tau} = \frac{d Q}{dt}
\end{equation}
where $Q$ is a property of the fluid, like vapor mixing ratio
or energy. Eq. \eqref{eq:material} gives  the rate of change of $Q$ for a tagged region of 
of fluid, with evolving spatial coordinates given by $x(a,b,c,\tau)$,
$y(a,b,c,\tau)$, $z(a,b,c,\tau)$.

Then using the chain rule we have:

\begin{equation}
  \label{eq:chain}
  \frac{\partial Q}{\partial \tau} = \frac{d Q}{dt} = 
\frac{\partial Q}{\partial t} \frac{\partial t}{\partial \tau} + 
\frac{\partial Q}{\partial x} \frac{\partial x}{\partial \tau} +
\frac{\partial Q}{\partial y} \frac{\partial y}{\partial \tau} +
\frac{\partial Q}{\partial z} \frac{\partial z}{\partial \tau}
\end{equation}
But the velocity of the infinitesimal volume is by definition:

\begin{equation}
  \label{eq:velocity}
  \mathbf{v} \equiv 
\left ( \frac{\partial x}{\partial \tau} , \frac{\partial y}{\partial \tau}, 
   \frac{\partial z}{\partial \tau} \right ) =
\left ( u, v, w \right )
\end{equation}
So we can rewrite (\ref{eq:chain}) as:

\begin{equation}
  \label{eq:chain2}
  \frac{\partial Q}{\partial \tau} = \frac{d Q}{dt} = 
\frac{\partial Q}{\partial t} +
u \frac{\partial Q}{\partial x}  + 
v \frac{\partial Q}{\partial y} +
w \frac{\partial Q}{\partial y} =
\frac{\partial Q}{\partial t} +
\mathbf{v} \cdot \nabla Q
\end{equation}
Thus   $\mathbf{v} \equiv 
   \left ( \frac{\partial x}{\partial \tau} , \frac{\partial y}{\partial \tau}, 
   \frac{\partial z}{\partial \tau} \right )$
makes explicit the idea that $\frac{\delta r }{\delta t}$ is the velocity
``following the flow''.


Then Taylor's frozen turbulence hypothesis is equivalent to the statement that:

\begin{equation}
  \label{eq:taylor}
\frac{\partial Q}{\partial \tau} =  \frac{d Q}{dt} = 0
\end{equation}
i.e., there are no sources or sinks of $Q$ changing the properties of the
fluid in the infinitesimal box.    In particular, the turbulence is
\textit{frozen}, and is not able to mix adjacent boxes on the time and
space scales of interest in the problem.  This doesn't mean that
there are no fluctuations, it just means that the \textit{statistics} of 
the fluctuations are stationary.  (Note that
the idea of frozen turbulence is due to 
\href{http://en.wikipedia.org/wiki/Geoffrey_Ingram_Taylor}{G. I. Taylor} (b.~1886, d.~1075)
not \href{http://en.wikipedia.org/wiki/Brook_Taylor}{Brook Taylor} (b.~1685, d.~1731). 


\end{document}
%%% Local Variables:
%%% mode: latex
%%% TeX-master: t
%%% End:
