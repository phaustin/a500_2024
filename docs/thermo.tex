\documentclass[12pt]{article}
\usepackage{geometry,fancyhdr,xr,hyperref,ifpdf,amsmath,rcs,indentfirst}
\usepackage{lastpage,longtable,Ventry,url,paunits,shortcuts,smallsec,color,tightlist,float}
\geometry{letterpaper,top=50pt,hmargin={20mm,20mm},headheight=15pt} 
\usepackage[stable]{footmisc}

\pagestyle{fancy} 

\RCS $Revision: 1.5 $
\RCS $Date: 2002/01/09 03:50:54 $

\fancypagestyle{first}{
\lhead{Liquid water static energy}
\chead{}
\rhead{page~\thepage/\pageref{LastPage}}
\lfoot{} 
\cfoot{} 
\rfoot{}
}

\ifpdf
    \usepackage[pdftex]{graphicx} 
    \usepackage{hyperref}
    \pdfcompresslevel=0
    \DeclareGraphicsExtensions{.pdf,.jpg,.mps,.png}
\else
    \usepackage{hyperref}
    \usepackage[dvips]{graphicx}
    \DeclareGraphicsRule{.eps.gz}{eps}{.eps.bb}{`gzip -d #1}
    \DeclareGraphicsExtensions{.eps,.eps.gz}
\fi

\begin{document}
\pagestyle{first}


\section{Thermodynamics}
\label{sec:thermodynamics}

It's a good idea to work with thermodynamic variables that are conserved under
a known set of conditions, since they can act as passive tracers and provide
information about the physical processes acting on the domain.  In particular,
we have available:

\begin{enumerate*}
  \item \textit{Conservation of mass}
  \item \textit{Conservation of momentum (newton's 2nd law)}
  \item \textit{Conservation of energy (first law)}
  \item \textit{Conservation of entropy (second law)}
\end{enumerate*}

The conditions we're interested in are horizontal and vertical
displacement on time scales that are too short to be affected by
radiation.  



\subsection{Conservation of mass}
\label{sec:conservation-mass}

Starting with conservation of mass, we assume that in the absence
of precip/evaporation the
\textit{total water mixing ratio}, defined as $q_T$ = kg ($H_2O$)/kg
(moist air), is conserved as a parcel moves, because while the volume,
pressure and temperature all change, the mass remains constant (in the
absence of mixing in regions with varying $r_v$), so:

\begin{equation}
  \label{eq:masscons}
  \frac{d q_T}{dt} = 0
\end{equation}
if mass is conserved and there is no condensation evaporation.

\subsection{Conservation of energy}
\label{sec:conservation-energy}

What about a conserved energy variable?  A good choice is the \textit{dry static energy},
$h_d$, which is conserved for adiabatic motion in a hydrostatic atmosphere.
To see why this is true, recall the definition of the first law:


\begin{equation}
  \label{eq:firstwh}
  \frac{du}{dt} = q - w
\end{equation}
where $u$ is the specific internal energy (\jkg) and $q$ and $w$ (\jkgs) are respectively the
rates at which the parcel is heated and work is done \textit{by} the parcel.
In all the problems in this course, we will only be concerned with the pressure-volume
working rate, which is defined by:

\begin{equation}
  \label{eq:pvwork}
  w = p \frac{d \alpha}{dt}
\end{equation}
where $\alpha$  (\volkg), the specific volume, is $1/density$ = $1/\rho$ .

It's not often that we get to work with constant volume situations in
the atmosphere.  More useful than the internal energy in atmospheric
thermodynamics is the \textit{specific enthalpy}, $h\ 
(\un{J\,kg^{-1}})$, defined as:

\begin{equation}
  \label{eq:enthalpy}
  h  = u + p\alpha
\end{equation}
Note that the $p \alpha$ term in (\ref{eq:enthalpy}) looks like pressure-volume
work, and can be interpreted as the work required at constant pressure $p$
to make room for the system with internal energy $u$\footnote{Schroeder, D. V., An Introduction
to Thermal Physics,\ Addison-Wesley, p.~149}.

Using $h$ we can rewrite the first law ($du/dt = q - p d\alpha/dt$) as

\begin{equation}
  \label{eq:entba}
  \frac{dh}{dt} = q   + \alpha \frac{dp}{dt}
\end{equation}

Taking $dp/dt=0$ in (\ref{eq:entba}), we can say that the heating rate
$q$ is equal to the time rate of change of enthalpy at constant
pressure.  Since evaporation and condensation occur at constant
pressure we will find $h$ useful when we deal with phase change.

Now expand $h$ in $p,T$ using the chain rule:

\begin{equation}
  \label{eq:Hchain}
  \frac{dh}{dt} = \frac{\partial h}{\partial T} \frac{dT}{dt} + 
                  \frac{\partial h}{\partial p} \frac{dp}{dt}
\end{equation}

and substitute (\ref{eq:Hchain}) into (\ref{eq:entba}) to get:

\begin{equation}
  \label{eq:firstH}
  q = \frac{\partial h}{\partial T} \frac{dT}{dt} + \left ( \frac{\partial h}{\partial p} - \alpha 
                        \right ) \frac{dp}{dt}
\end{equation}


If the pressure is constant, then $dp/dt$=0, and (\ref{eq:firstH})
produces the definition of the \textit{heat capacity at constant
  pressure}:

\begin{eqnarray}
  \label{eq:hcp}
  c_p &=& \frac{\partial{h}}{\partial T}\\
  q &=& c_p \frac{\partial T}{\partial t}\ \mathrm{at\ constant\ pressure}
\end{eqnarray}
For dry air $c_{pd} = 1004\ \jkgk$.

Also, thermodynamics tells us that over the range of atmospheric
temperatures and pressure $\partial h/\partial p \approx 0$, so using
\eqref{eq:firstH} and and \eqref{eq:entba} we can get:

\begin{equation}
  \label{eq:enthal}
  \frac{dh}{dt} = c_p \frac{dT}{dt}
\end{equation}
since $c_p$ depends only very weakly on pressure.

Now what about $dp/dt$ in \eqref{eq:entba} for the case of vertical
motion?  For the boundary layer, it is a very good approximation to
assume \textit{hydrostatic balance}, which is the statement that an
the vertical pressure differential across a layer provides exactly the
amount of force necessary to balance gravity:

  \begin{figure}[H]
    \begin{center}
       \input hydro.pstex_t
      \caption{hydrostatic balance for a $1 \times dz\  \un{m^3}$ layer}
      \label{fig:hydro}
    \end{center}
  \end{figure}

\vspace{0.1in}

In symbols, the balance shown by Figure \ref{fig:hydro} implies that:

\begin{equation}
  \label{eq:hydro}
  \frac{dp}{dt} = - \rho g \frac{dz}{dt}
\end{equation}
\textit{Question:  is this the same pressure p as the local pressure given by the equation of state: $p=\rho R_d T$?}

Substituting \eqref{eq:hydro} into \eqref{eq:entba} and recognizing that
$\alpha \rho$=1, the first law in the case of hydrostatic balance becomes:

\begin{equation}
  \label{eq:first2}
  \frac{dh}{dt} = q - g \frac{dz}{dt}
\end{equation}

In particular, if the heating rate $q$=0, then:

\begin{equation}
  \label{eq:first3}
  \frac{dh}{dt} +  g \frac{dz}{dt} = \frac{d h_d}{dt} =  0
\end{equation}
defines a conservation equation for the \textit{dry static energy} $h_d$, where:

\begin{equation}
  \label{eq:hd}
  h_d = h + gz
\end{equation}


This conservation makes $h_d$
a useful tracer, since changes to $h_d$ have to come from some
diabatic process (radiation, evaporation), and can't come from simply
lifting or sinking air.


Note also that $gz$ is just the \textit{potential energy per
  kilogram}.  So as the parcel decreases or increases its altitude,
energy is moving between potential and internal, while $h_d$ is
staying the same.

We can also take \eqref{eq:first3} apart to get an equation for the
\textit{dry adiabatic lapse rate}:

\begin{subequations}
\begin{eqnarray}
  \label{eq:drylapse}
\frac{dh_d}{dt} &=&   c_p \frac{dT}{dt} -g \frac{dz}{dt} = 0\\
\mathrm{so}\ \frac{dT}{dz} &=& \Gamma_d = -\frac{g}{c_p} \approx -9.8\ \un{K\,{km^{-1}}}
\end{eqnarray}
\end{subequations}
which is the rate at which internal energy is converted to
potential energy during adiabatic ascent.


\subsection{Conservation of entropy}
\label{sec:conservation-entropy}

First remember that the 2nd law of thermodynamics relates the entropy $s$ to the heating rate $q$
and temperature $T$:

\begin{equation}
  \label{eq:second}
  \frac{ds}{dt} \geq \frac{q}{T}
\end{equation}
where the $>$ sign holds when the process is \textit{irreversible}.

Also note that the equation of state for dry air is given by:

\begin{equation}
  \label{eq:eos}
  p = \rho R_d T
\end{equation}
where $R_d$ (287 \jkgk) is the gas constant for dry air.

So divide \eqref{eq:entba} by $T$ and  rewrite the first law  as:

\begin{equation}
  \label{eq:poten1}
  \frac{q}{T}  = \frac{1}{T} \left ( \frac{dh}{dt} - \alpha \frac{dp}{dt} \right ) =  
    \frac{c_p}{T} \frac{dT}{dt} - \frac{R_d T}{T\, p} \frac{dp}{dt} = 
    \frac{c_p}{T} \frac{dT}{dt} - \frac{R_d }{ p} \frac{dp}{dt} \leq \frac{ds}{dt}
\end{equation}
where I've also included invoked the second law to relate this to $s$.

Again, if the process is  adiabatic then $q=0$, 

\begin{subequations}
\begin{eqnarray}
  \label{eq:adiathet}
  c_p \frac{1}{T} \frac{dT}{dt} &=& \frac{R_d}{p} \frac{dp}{dt}\\
  c_p  d\log{T} &=& R_d d \log {p}\label{eq:adiathetb}
\end{eqnarray}
\end{subequations}

 Integrating both sides of (\ref{eq:adiathetb}) from the
surface, with (temperature, pressure) given by $(\theta, p_0)$
to a lower pressure $p$ and lower (why?) temperature $T$
gives  \textit{Poisson's equation}:

\begin{equation}
  \label{eq:poisson}
  \frac{c_p}{R_d} \log \left ( \frac{T}{\theta} \right ) = \log \left ( \frac{p}{p_0} \right )
\end{equation}

Equation (\ref{eq:poisson}) defines the \textit{potential temperature}, $\theta$
which is conserved for adiabatic ascent/descent  (BLM 1.5.1c):

\begin{equation}
  \label{eq:pottemp}
  \theta =  T \left ( \frac{p_0}{p} \right )^{R_d/c_p} =
 T \left ( \frac{p_0}{p} \right )^{(c_p - c_v)/c_p} =  
 T \left ( \frac{p_0}{p} \right )^{\frac{\gamma - 1}{\gamma}} 
\end{equation}
where $R_d/c_p = 287/1004 =0.286$ and $\gamma = c_p/c_v$.  

Like $h_d$, $\theta$ is constant for a dry adiabatic process. To see this
first  differentiate \eqref{eq:pottemp} to get:

\begin{equation}
  \label{eq:differ}
\frac{c_p}{\theta} \frac{d \theta} {dt} =  \frac{c_p}{T} \frac{dT}{dt} - \frac{R_d }{ p} \frac{dp}{dt}
\end{equation}

Then use \eqref{eq:poten1} to show that:

\begin{equation}
  \label{eq:thetaIsS}
  \frac{c_p}{\theta} \frac{d \theta} {dt} = \frac{ds} {dt} \geq \frac{q}{T}
\end{equation}
and if $q$=0 and the process is reversible, both $\theta$ and $s$ are constant.

This means that there's a one to one relationship between $\theta$ and the
entropy $s$:

\begin{equation}
  \label{eq:entrop1}
  ds = c_p \frac{d \theta}{\theta}
\end{equation}

or 

\begin{equation}
  \label{eq:entrop2}
  s \propto  c_p \log \theta
\end{equation}

\section{Liquid water static energy}
\label{sec:liquid-water-static}


Once we introduce water vapor and liquid water, we're dealing
with a new mixture that can't be described by the ideal gas equation.
It does have an enthalpy $H$ (J):

\begin{equation}
  \label{eq:hmix}
  H=m_d h_d + m_v h_v + m_l h_l
\end{equation}
where $m$ stands for mass, $H$ is the total enthalpy (J) and $h_x$ 
(\un{J\,kg^{-1}}) is the specific enthalpy where
subscripts $x=(d,v,l)$ respectively denote dry air, water vapor
and liquid water. 

We will assume that the mass of dry air stays fixed, and
write the specific enthalpy of the mixture as:

\begin{equation}
  \label{eq:mix}
  h = H/m_d\ \un{J\,kg^{-1}}
\end{equation}

Note that the total entropy still obeys the first law:

\begin{equation}
  \label{eq:first}
  q\,dt= dh - \alpha dp
\end{equation}
But what is $dh$?  Calculus says:

\begin{equation}
  \label{eq:Hchain}
  dH = \frac{\partial H}{\partial T} dT + 
       \frac{\partial H}{\partial p} dp +
       \frac{\partial H}{\partial m_v} d m_v + 
       \frac{\partial H}{\partial m_l} d m_l
\end{equation}

And by the definition of the heat capacity:

\begin{equation}
  \label{eq:partialT}
  \frac{\partial H}{\partial T} = m_d c_{pd} + m_v c_{pv} + m_l c_l
\end{equation}
Which, after dividing through by $m_d$, produces the heat
capacity of the mixture, $c_p\ (\un{J\,kg^{-1}\,K^{-1}})$:

\begin{equation}
  \label{eq:partialTspec}
  \frac{\partial h}{\partial T} =  c_{pd} + r_v c_{pv} + r_l c_l = 
      c_p
\end{equation}

The second term in (\ref{eq:Hchain}),
$\frac{\partial H}{\partial p}=0$ to an excellent approximation
(remember that $H=U + PV$), and looking at (\ref{eq:hmix}) it's
apparent that $\frac{\partial H}{\partial m_v}=h_v$ and
$\frac{\partial H}{\partial m_l}=h_l$, yielding a new version of
(\ref{eq:Hchain}):

\begin{equation}
  \label{eq:newH}
  dh = c_p dT + 
       h_v d r_v +
       h_l d r_l
\end{equation}
We'll start with the simplest scenario: the total water mass is
conserved and the mixture is at its saturation mixing ratio
$r_{sat} = \epsilon e_s(T)/(p - e_s(T))$.  Then the total water mass
$m_t=m_v + m_l$  satisfies 
$d m_t = 0 = d m_v + d m_l = d m_s + dm_l = d r_{sat} + d r_l$, so
that $dr_v = d r_{sat} = - d r_l$ and (\ref{eq:newH}) becomes:

\begin{equation}
  \label{eq:newHsat}
  dh = c_p dT + 
       (h_v - h_l) d r_{sat} = c_p dT + l_v d r_{sat}
\end{equation}
where $l_v\ (\un{J\,kg^{-1}}) = h_v - h_l$ 
is called the \textit{enthalpy of evaporation}
or the \textit{latent heat}.  It represents the energy needed to
break a liquid water molecule away from its neighbors and send it
into  the vapor phase.  It takes about 2.5 
$\times 10^{6}\ \un{J}$ to evaporate a kg of water at
0 \degc.

Note that because the mass of water is conserved, I could just as well write (\ref{eq:newHsat}) as:

\begin{equation}
  \label{eq:newHsatb}
  dh = c_p dT  - l_v dr_l
\end{equation}


I could also go through the steps above for a more general  mixture of droplets, rain and ice
and get the expression for enthalpy used in Khairoutdinov and Randall:


\begin{equation}
  \label{eq:newHsatKR}
  dh = c_p dT + 
        c_p dT - l_v (dr_l + dr_r) - l_s (dr_i + dr_s + dr_g)
\end{equation}

Since $dh$ is still the enthalpy, it has to obey the first law:

\begin{equation}
  \label{eq:newHsatKR2}
  \frac{dh}{dt} = q - \alpha \frac{dp}{dt} 
\end{equation}
or if we have hydrostatic balance:

\begin{equation}
  \label{eq:newHsatKR3}
  \frac{dh}{dt} + g\frac{dz}{dt}  = q
\end{equation}

Note that (\ref{eq:newHsatKR3}) can be rearranged to give the saturated moist adiabatic lapse rate
with q=0:

\begin{equation}
  \label{eq:newHsatKR3b}
ds_v =  dh + g dz   = c_p dT + l_v d r_{sat} + g dz = 0
\end{equation}
Equation (\ref{eq:newHsatKR3}b) defines the \textit{moist static energy}, $s_v$,
which is conserved for adiabatic ascent.  Again, since total water is conserved,
we are also free to define the \textit{liquid water static energy, $s_l$}:

\begin{equation}
  \label{eq:newHsatKR3c}
ds_l =  dh + g dz   = c_p dT - l_v d r_l + g dz = 0
\end{equation}
Integrating (\ref{eq:newHsatKR3c}) gives
\begin{equation}
  \label{eq:newHsatKR3d}
s_l = c_p T - l_v r_l + g z
\end{equation}

How much does condensation heating change the dry adiabatic lapse rate?
We can calculate this:

\begin{equation}
  \label{eq:newHsatKR4}
  dh + g dz  = c_p dT + l_v d r_{sat} + g dz = 0
\end{equation}

\begin{equation}
  \label{eq:newHsatKR5}
  \frac{dh}{dz}= c_p \frac{dT}{dz} + l_v \frac{d r_{sat}}{dz} =  -g
\end{equation}

\begin{equation}
  \label{eq:newHsatKR6}
  \frac{dh}{dz}= c_p \frac{dT}{dz} + l_v \frac{d r_{sat}}{dT} \frac{dT}{dz} =  -g
\end{equation}


\begin{equation}
  \label{eq:newHsatKR6b}
 \frac{dT}{dz} = - \left ( \frac{g}{c_p}  \right ) \left ( \frac{1}{1 + l_v \frac{d r_{sat}}{dT}} 
\right ) 
\end{equation}

For typical midlatitude conditions, the evaporation and condensation of water reduces the
magnitude of $dT/dz$ from about -10 \un{K/km} to -6 \un{K/km}.

\section{Moist entropy}
\label{sec:moist-entropy}


We can divide (\ref{eq:first}) by $T$ and
substitute (\ref{eq:newH}) for $dh$ to get
the entropy of the mixture:

\begin{equation}
  \label{eq:s}
ds = \frac{q dt}{T} = \frac{c_p}{T} dT + \frac{l_v}{T} d r_{sat} - 
\frac{R_d}{p} dp
\end{equation}
Note that we have made the approximation that $\alpha$, the specific density,
is about the same for moist and dry air.  You'll integrate (\ref{eq:s})
numerically using python, but for now we can make some progress by
using the approximation that

\begin{equation}
  \label{eq:lvapprx}
  \frac{l_v}{T} d r_{sat} \approx d \left ( \frac{l_v r_{sat}}{ T} \right )
\end{equation}
(note that this is not a good approximation (errors of about 30\% at high temperature,
but fortuitously there's cancellation in the neglected terms that makes it nearly exact, we'll
derive the full version later in the course)


We'll also need the result from the entropy notes  that:

\begin{equation}
  \label{eq:theta}
  c_p \frac{d\theta}{\theta} = c_p \frac{dT}{T} - R_d\frac{dp}{p}
\end{equation}
Substituting (\ref{eq:theta}) and (\ref{eq:lvapprx}) into
(\ref{eq:s}), and taking the adiabatic case ($q=0$) gives a
a differential equation for the potential temperature under
an adiabatic transformation:

\begin{equation}
\label{eq:thetaode}
c_p \frac{ d \theta}{\theta} = - \frac{l_v}{T} d r_{sat} \approx - d \left ( \frac{l_v r_{sat}}{ T} \right )  
\end{equation}

As usual, I can fatten and thin (\ref{eq:thetaode}) to get an equation
for the change in $\theta$ wrt, say, height, for an adiabatic parcel:

\begin{equation}
\label{eq:dheight}
 \frac{ d \log \theta}{dz} = - \frac{l_v}{c_ p T}\frac{ d r_{sat}}{dz}
\end{equation}
\textit{Convince yourself that (\ref{eq:dheight}) implies that
  $\theta$ increases with height along a moist adiabat}.  We can
integrate (\ref{eq:dheight}) to give $\theta$ as a function of height
(or equivalently, pressure or time).  Since
$dr_l$ = $-dr_{sat}$, (\ref{eq:dheight}) is equivalent to:

\begin{equation}
\label{eq:dheightl}
 \frac{ d \log \theta}{dz} = \frac{l_v}{c_ p T}\frac{d r_l}{dz}
\end{equation}

There are quite a few different ways to label (\ref{eq:dheight}) or 
(\ref{eq:dheightl}) but they all provide different names for the
same adiabat.  In this course we'll use either the 
\textit{equivalent potential temperature} $\theta_e$, the 
\textit{liquid water potential temperature} $\theta_l$ or the
\textit{wet bulb potential temperature} $\theta_w$.  To find the
\thetae label use (\ref{eq:lvapprx}) and integrate the approximate
equation from a height where the temperature is so cold that
effectively $r_{sat}^\prime$ = 0, down to the height of interest, where
$r_{sat}^\prime = r_w$:

\begin{equation}
  \label{eq:thetae0}
  \int_{\thetae}^\theta d \log \theta^\prime = 
\int_0^{r_{sat}} d \left ( - \frac{l_v r_{sat}^\prime}{c_p T} \right )
\end{equation}

Since both sides are perfect differentials we can immediately write
down (\ref{eq:thetae0}) as:

\begin{equation}
  \label{eq:thetae1}
  \log \left ( \frac{\theta}{\thetae} \right ) = -\frac{l_v r_{sat}}{c_p T} 
\end{equation}

or:

\begin{equation}
  \label{eq:thetae2}
  \thetae = \theta \exp \left ( \frac{l_v r_{sat}}{c_p T} \right )
\end{equation}

I'll leave it to you to show by the same approach that (\ref{eq:dheightl}) can
be integrated to give:

\begin{equation}
  \label{eq:thetal}
  \thetal = \theta \exp \left ( - \frac{l_v r_l}{c_p T} \right )
\end{equation}

One especially useful feature of \thetae is that it is approximately
constant for precipitation parcels, because $r_l$ doesn't appear in
(\ref{eq:s}), except as a small term in the mixture heat capacity $c_p$.
In contrast \thetal assumes implicitly that there is no precipitation
along the vertical integration, since we assume $dr_l = dr_{sat}$ everywhere.
The advantage of $\theta_l$ is that it reduces to $\theta$ when
liquid water is zero (say below or outside of cloud).



\end{document}


%%% Local Variables:
%%% mode: latex
%%% TeX-master: t
%%% End:
