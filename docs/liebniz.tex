\documentclass[12pt]{article}
\usepackage{xr,geometry,fancyhdr,hyperref,ifpdf,rcs,amsmath}
\usepackage{color,lastpage,longtable,Ventry,url,paunits,shortcuts,smallsec}
\usepackage{Ventry,smallsec}
\geometry{letterpaper,top=50pt,hmargin={20mm,20mm},headheight=15pt} 

\pagestyle{fancy} 
\fancypagestyle{first}{
\lhead{\textbf{Liebniz' Rule}}
\chead{ATSC 500} 
\rhead{p.~\thepage/\pageref{LastPage}}
\lfoot{} 
\cfoot{} 
\rfoot{}
}

\ifpdf
    \usepackage[pdftex]{graphicx} 
    \usepackage{hyperref}
    \pdfcompresslevel=0
    \DeclareGraphicsExtensions{.pdf,.jpg,.mps,.png}
\else
    \usepackage{hyperref}
    \usepackage[dvips]{graphicx}
    \DeclareGraphicsRule{.eps.gz}{eps}{.eps.bb}{`gzip -d #1}
    \DeclareGraphicsExtensions{.eps,.eps.gz}
\fi

\newcommand{\rad}{%
   \ensuremath{I}
}

\newcommand{\irrad}{%
   \ensuremath{F}
}

\newcommand{\optd}{%
   \ensuremath{\tau}
}

\newcommand{\zenith}{%
   \ensuremath{\theta}
}

\newcommand{\azimuth}{%
   \ensuremath{\phi}
}

\newcommand{\trans}{%
   \ensuremath{t}
}

\newcommand{\abs}{%
   \ensuremath{\alpha}
}


\newcommand{\volabs}{%
   \ensuremath{\sigma}
}

\newcommand{\massabs}{%
   \ensuremath{\kappa}
}


\begin{document}
\pagestyle{first}

\begin{center}
ATSC 500:  Liebniz' rule\\
\end{center}

Quick review of Liebniz rule for discussion of Stull Section 7.4.3, p. 271
(Also recall the
\href{http://clouds.eos.ubc.ca/~phil/courses/atsc500/docs/stationarity.pdf}{Stationarity notes} from Lecture 2 and Stull 
\href{http://clouds.eos.ubc.ca/~phil/courses/atsc500/docs/readings/chap2.pdf}{Chapter 2}, p. 39 eq. 2.4.2h


\section{Liebniz' Rule}
\label{sec:liebniz}

The fundamental theorem of calculus says that:

If 

\begin{equation}
  \label{eq:fund}
  f(x) = \frac{dF}{dx}
\end{equation}

then

\begin{equation}
  \label{eq:fund2}
  F(x) = \int f(x) dx
\end{equation}

But what about definite integrals?  These are just numbers, unless the limits are themselves variables:

\begin{equation}
  \label{eq:examp1}
  \int_1^3 x^2 dx = \frac{3^3}{3} - \frac{1^3}{3} = 8 \frac{2}{3}
\end{equation}

\begin{equation}
  \label{eq:examp2}
  \int_0^s x^2 dx = \frac{s^3}{3} - \frac{0^3}{3} = \frac{s^3}{3}
\end{equation}

Liebniz' rule says:

\begin{equation}
  \label{eq:lieb1}
  \frac{d}{ds} \int_0^s f(x) dx = f(s)
\end{equation}

\begin{equation}
  \label{eq:lieb2}
  \frac{d}{ds} \int_s^\infty f(x) dx = -f(s)
\end{equation}

Does this work for our specific example?

\begin{equation}
  \label{eq:specif}
  \frac{d}{ds} \int_0^s x^2 dx = \frac{d}{ds} \frac{s^3}{3} = s^2 =f(s)
\end{equation}

So yes, it does.  Convince yourself that (\ref{eq:lieb2}) also works in a specific case.

For something like total integrated water vapor W ( \un{kg\,m^{-2}} ) this looks 
like

\begin{equation}
  \label{eq:taudown}
\frac{d W}{dz} = \frac{d}{dz} \int_0^z \rho_{air} r_v dz =  \rho_{air} (z)  q_v(z)
\end{equation}
where $q_v$ is the vapor mixing ratio in kg/kg and $\rho_{air}$ is the
moist air density.

\end{document}

%%% Local Variables:
%%% mode: latex
%%% TeX-master: t
%%% End:
