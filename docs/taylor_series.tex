\documentclass[12pt]{article}
\usepackage{geometry,fancyhdr,xr,hyperref,ifpdf,amsmath,rcs,indentfirst}
\usepackage{lastpage,longtable,Ventry,url,paunits,shortcuts,smallsec,color,tightlist,float}
\geometry{letterpaper,top=50pt,hmargin={20mm,20mm},headheight=15pt} 
\usepackage[stable]{footmisc}
\pagestyle{fancy} 

\RCS $Revision: 1.5 $
\RCS $Date: 2002/01/09 03:50:54 $

\fancypagestyle{first}{
\lhead{Taylor's series vs. Reynold's decomposition}
\chead{}
\rhead{page~\thepage/\pageref{LastPage}}
\lfoot{} 
\cfoot{} 
\rfoot{}
}

\ifpdf
    \usepackage[pdftex]{graphicx} 
    \usepackage{hyperref}
    \pdfcompresslevel=0
    \DeclareGraphicsExtensions{.pdf,.jpg,.mps,.png}
\else
    \usepackage{hyperref}
    \usepackage[dvips]{graphicx}
    \DeclareGraphicsRule{.eps.gz}{eps}{.eps.bb}{`gzip -d #1}
    \DeclareGraphicsExtensions{.eps,.eps.gz}
\fi


\begin{document}
\newcommand{\vect}[1]{\boldsymbol{\vec{#1}}}
\pagestyle{first}

\section{Expanding the equation of state -- Taylor's series}


Let's compare two different ways to calculate perturbations --  expand the equation of state via a Taylor's
series expansion or Reynold's decomposition.  Specifically consider this equation from the
\href{http://en.wikipedia.org/wiki/Taylor_series}{Wikipedia entry on Taylor series}:

\begin{align}
  \label{eq:taylor}
f(x,y) & \approx f(a,b) +(x-a)\, f_x(a,b) +(y-b)\, f_y(a,b)\nonumber \\
& {}\quad + \frac{1}{2!}\left[ (x-a)^2\,f_{xx}(a,b) + 2(x-a)(y-b)\,f_{xy}(a,b) +(y-b)^2\, f_{yy}(a,b) \right]
\end{align}
where $f_{xy} = \frac{ \partial^2 f}{\partial x \partial y }$, etc.  The goal
of this problem is to expand $f=p=\rho R_d T$ about the point 
$p_0(z) = \rho_0(z) R_d T_0(z)$ where $p_0,\ \rho_0,\ T_0$ are the pressure,
density and temperature at height $z$ for a hydrostatic atmosphere.  Using (\ref{eq:taylor}) with $a=\rho_0$ and $b=T_0$ you should be able to show that
to second order:

\begin{equation}
  \label{eq:delta}
\frac{\Delta p }{p_0}    = \frac{ \Delta T}{T_0} + \frac{ \Delta \rho}{\rho_0} 
 + \frac{ \Delta T \Delta \rho}{T_0 \rho_0} 
\end{equation}
where $\Delta p = p - p_0$, $\Delta T = T - T_0$, $\Delta \rho = \rho - \rho_0$

Note that $\Delta p$, $\Delta T$, and $\Delta \rho$ are all functions of
(t,x,y,z).

If the atmosphere is close to to hydrostatic balance, then we can neglect the second order 
$\frac{ \Delta T \Delta \rho}{T_0 \rho_0} $ term and write

\begin{equation}
  \label{eq:delta}
\frac{\Delta p }{p_0}    = \frac{ \Delta T}{T_0} + \frac{ \Delta \rho}{\rho_0} 
\end{equation}


\section{Expanding the equation of state -- Reynold's decomposition}

Compare what you just did in Equation~(\ref{eq:delta}) with a Reynold's decomposition:

Define:

\begin{subequations}
\begin{align}
  \label{eq:pbar}
  \langle p \rangle & = \int_{ 0}^{\infty} f(p) p \!\,d p\\
  p^\prime& = p - \langle p \rangle \\
  \langle {T \,\rho } \rangle & = \int_{ 0}^{\infty} \int_{ 0}^{\infty} f(T, \rho) \; T \; \rho \; \,dT d \rho \label{eq:corel}
\end{align}
\end{subequations}
where is $f$ is the pdf.

Substituting $T = \langle T \rangle + T^\prime$ and $\rho = \langle {\rho} \rangle + \rho^\prime$ into (\ref{eq:corel})
and using the definition of probability averaging gives:

\begin{subequations}
\begin{align}
  \label{eq:p}
  p &= R_d \left [ \langle {\rho} \rangle \; \langle {T} \rangle  + \langle {\rho} \rangle  T^\prime + \langle {T} \rangle  \rho^\prime + \rho^\prime T^\prime \right ] \\
\langle {p} \rangle  & = R_d \left [ \langle {\rho} \rangle  \; \langle {T} \rangle  + \langle {\rho^\prime T^\prime} \rangle  \right ] 
\end{align}
\end{subequations}

If the density and temperature fluctuations were uncorrelated, then $\langle {\rho^\prime T^\prime} \rangle \approx 0$ and
$\langle {p} \rangle  = R_d \langle {\rho} \rangle  \langle {T} \rangle $ and 


\begin{subequations}
\begin{align}
  \label{eq:pbar}
  p^\prime& = p - \langle {p} \rangle  \approx R_d \left [  \langle {\rho} \rangle  T^\prime + \langle {T} \rangle  \rho^\prime + \rho^\prime T^\prime \right ] \\
\frac{p^\prime}{\langle {p} \rangle } & \approx \frac{ T^\prime}{ \langle {T} \rangle } + \frac{\rho^\prime }{\langle {\rho} \rangle } 
+\frac{T^\prime \rho^\prime }{ \langle {T} \rangle  \langle {\rho} \rangle }
\end{align}
\end{subequations}
Note that we haven't said anything about the size of $T^\prime$ or $\rho^\prime$, just their correlation.

\end{document}

%%% Local Variables:
%%% mode: latex
%%% TeX-master: t
%%% End:
