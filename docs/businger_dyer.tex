\documentclass[11pt]{article}
\usepackage{geometry,ifpdf,smallsec,amsmath}
\geometry{letterpaper,top=50pt,hmargin={20mm,20mm},headheight=15pt} 

\ifpdf
    \usepackage[pdftex]{graphicx} 
    \usepackage{hyperref}
    \pdfcompresslevel=0
    \DeclareGraphicsExtensions{.pdf,.jpg,.mps,.png}
\else
    \usepackage{hyperref}
    \usepackage[dvips]{graphicx}
    \DeclareGraphicsRule{.eps.gz}{eps}{.eps.bb}{`gzip -d #1}
    \DeclareGraphicsExtensions{.eps,.eps.gz}
\fi

\usepackage{fancyheadings}
\pagestyle{fancy} 
\lhead{ATSC500} 
\rhead{page~\thepage} 
\chead{Surface layer simlarity}
\lfoot{}
\cfoot{}
\rfoot{}

\def\lb{\left(}
\def\rb{\right)}
\def\LB{\left[}
\def\RB{\right]}
\def\Lb{\left\{}
\def\Rb{\right\}}
\def\la{\langle}
\def\ra{\rangle}

\def\dl{\partial}
\def\nb{\nabla}
\def\vp{\varphi}

\newcommand{\vect}[1]{\boldsymbol{\vec{#1}}}


\begin{document}

\section{Stability review -- Chapter 5}


\subsection{Richardson number}
\label{sec:richardson-number}

Stull section 5.6 gives us several definitions for
the Richardson number:


\begin{align}
  \label{eq:ri}
\text{gradient } Ri&= \frac{    \frac{ g }{\overline{\theta_v}} \left ( \frac{\partial \overline{  \theta_v} }{\partial z}  \right )  }%
{ \left [ \left ( \frac{ \partial \overline{ U}}{\partial z} \right )^2  + 
\left ( \frac{ \partial \overline{ V}}{\partial z} \right )^2  \right ]} \\
\text{flux }Ri_f&= \frac{    \frac{ g }{\overline{\theta_v}} \left ( \overline{ w^\prime \theta_v^\prime }   \right )  }%
{\left [ \left ( \overline{ w^\prime u^\prime } \right ) \frac{ \partial \overline{ U}}{\partial z}   + 
\left (\overline{ w^\prime v^\prime }\right ) \frac{ \partial \overline{ V}}{\partial z}   \right ]} \\
\text{bulk }Ri_B&=  \frac{ g }{\overline{\theta_v}} \frac{ \Delta \overline{\theta_v } \Delta z  }%
{ \left [ \left ( \Delta \overline{U } \right )^2 + \left ( \Delta \overline{V } \right )^2   \right ]}
\end{align}

Each of these is the \textit{\textbf{negative}} of the ratio of the buoyancy production/consumption
term in the TKE equation to the shear
production term (i.e.~ -(III)/IV in 5.1a on p.~152). Note the sign convention is such that 
a postive Richardson number is typically found in a stable layer where $M$ increases
with height away from the surface and the buoyancy flux is negative, so that TKE
would be decreasing if only Stull 5.1a terms III and IV are considered. 
If Ri falls below about 1, then there is net energy release either from
large shear (for positive Ri) or convection (for negative Ri)
and instabilities can move energy from the mean shear to the
perturbations.  

In Chapter 6 we will see that under a fairly broad set of circumstances we can use
``first order closure'' aka ``gradient diffusion'' or ``K-theory'' to approximate fluxes of entropy and momentum as:

\begin{eqnarray}
\overline{w^\prime \theta^\prime} &\approx K_H \frac{d\overline{\theta}}{dz}   \\
\overline{w^\prime u^\prime} & \approx K_M \frac{d\overline{U}}{dz}
\end{eqnarray}


Using first order closure the flux and
gradient Richardson numbers are related by:

\begin{equation}
  \label{eq:fluxgrad}
  R_f = (K_H/K_M) Ri
\end{equation}


\subsection{Obukhov length}
\label{sec:obukhov-length}

On p.~180 Stull introduces the Obukhov length L, which is also formed by taking the ratio
of the buoyancy to the shear terms in the TKE equation.  The major additional information
that is contained in $L$ is that there is a comparison with the shear profile that
would be expected from a neutral surface layer.   This is found by nondimensionalizing the
speed shear $dM/dz$ and the stability $d\theta/dz$ by the expected surface layer results, 
and using that to non-dimensionalize the ratio of the buoyancy to shear production terms.

First remember that from 
\href{http://clouds.eos.ubc.ca/~phil/courses/atsc500/docs/velocity_scales.pdf}{velocity scale notes} or
that in the neutral
surface layer where $u^\prime w^\prime$ is constant with height:

\begin{equation}
  \label{eq:shear}
  \frac{ \partial M}{\partial z} = \frac{ u_*}{kz} 
\end{equation}
So use this value of the neutral layer gradient for speed to non-dimensionalize
an observed $\frac{  \partial \overline{ M}  }{\partial z}$ profile in conditions
that differ from neutral stratification: (cf. Stull 9.7.1.c on p.~378):


\begin{equation}
  \label{eq:surface}
\phi_M  =   \frac{kz  }{u_*} \frac{  \partial \overline{ M}  }{\partial z}
\end{equation}
and for buoyancy:

\begin{equation}
  \label{eq:buoyancy}
\phi_H = \frac{k z}{\theta_*}  \frac{\partial  \overline{   \theta_v }    }{\partial z}  
\end{equation}
where $\theta_* =  \left (\overline{ w^\prime \theta_v^\prime } \right )/u_*$.  

In the surface layer first order closure gives:

\begin{gather}
  \label{eq:ktheory}
   \left (\overline{ w^\prime \theta_v^\prime } \right ) = K_H \frac{\partial \overline{ \theta_v} }{\partial z} = u_* \theta_*\\
   \left (\overline{ w^\prime u^\prime } \right ) = K_M \frac{\partial M }{\partial z} = u*^2
\end{gather}
and inserting (\ref{eq:surface}) and (\ref{eq:buoyancy}) into (\ref{eq:ktheory}) gives:

\begin{equation}
  \label{eq:ksurface}
  \frac{\phi_M }{\phi_H} = \frac{K_H }{K_M} 
\end{equation}

If we use $u_*^2 \frac{ \partial  M}{\partial z} $ as the neutral surface
layer value for $\left [ \left ( \overline{ w^\prime u^\prime } \right ) \frac{ \partial \overline{ U}}{\partial z}   + 
\left (\overline{ w^\prime v^\prime }\right ) \frac{ \partial \overline{ V}}{\partial z}   \right ]$
and put in (\ref{eq:shear}) for neutral surface layer $\frac{ \partial  M}{\partial z}$, then the
ratio of the buoyancy production to the shear production in the neutral surface layer
becomes (Stull p. 181):

\begin{multline}
  \label{eq:rinew}
\frac{   - \frac{ g }{\overline{\theta_v}} \left ( \overline{ w^\prime \theta_v^\prime }   \right )  }%
{ \left [ \left ( \overline{ w^\prime u^\prime } \right ) \frac{ \partial \overline{ U}}{\partial z}   + 
\left (\overline{ w^\prime v^\prime }\right ) \frac{ \partial \overline{ V}}{\partial z}   \right ]}
= \frac{  -  \frac{ g }{\overline{\theta_v}} \left ( \overline{ w^\prime \theta_v^\prime }   \right )  }%
{u_*^2 \frac{ \partial  M}{\partial z}} =\\
 \frac{  -  \frac{ g }{\overline{\theta_v}} \left ( \overline{ w^\prime \theta_v^\prime }   \right )  }%
{u_*^2 \frac{u_* }{kz}}  = z \frac{- k g \left ( \overline{ w^\prime \theta_v^\prime }   \right ) }%
{\overline{\theta_v} u_*^3} =
  z \left ( -\frac{k g \left ( \overline{ w^\prime \theta_v^\prime }   \right ) }%
{\overline{\theta_v} u_*^3} \right ) 
=  \frac{ z}{L}  = \zeta
\end{multline}
where the negative sign in the numerator gives $\zeta$ the same sign as $Ri$, so that positive
Obukhov length $L$, where

\begin{equation}
  \label{eq:ol}
  L = \frac{-\overline{\theta_v} u_*^3 }{k g \overline{  w^\prime \theta_v^\prime }} 
\end{equation}
indicates a negative buoyancy flux  we
a stable surface layer.  Given the definition of the convective velocity scale
in the \href{http://clouds.eos.ubc.ca/~phil/courses/atsc500/docs/velocity_scales.pdf}{velocity scale notes}:

\begin{equation}
  \label{eq:wstar}
  w_* = \left (  \frac{ g z_i}{\overline{ \theta_v}} \left (  \overline{ w^\prime \theta_v^\prime } \right )_s \right )^{1/3}
\end{equation}

We can set $z=z_i$ in (\ref{eq:ol}) and get:

\begin{equation}
  \label{eq:mo}
  \frac{ z_i}{L} = -k \frac{ w^3_*}{u_*^3} 
\end{equation}

\textbf{Material below here applieds to chapter 7, p. 267 and Chapter 9}

\subsection{Relating $Ri$ and $L$}
\label{sec:relating-ri-l}

We can write  a general relationship between $\zeta$ and $Ri$ in the non-neutral surface
layer given the Bussinger-Dyer relations of Stull Section 9.6.1.  
Inserting  (\ref{eq:surface}) and (\ref{eq:buoyancy}) into (\ref{eq:ri}) and using
(\ref{eq:ol}) and (\ref{eq:ktheory}) gives:

\begin{equation}
  \label{eq:ri-mo}
  Ri = \zeta \frac{ \phi_H}{\phi_M^2} 
\end{equation}

The suface layer fits are given in Stull p.~369.  The typical way to present them
(e.g.~Garrett 3.33)

\begin{equation}
  \label{eq:busdy1}
\phi_M = 
\begin{cases}
   ( 1 - \gamma_1 \zeta )^{-1/4}  &  \text{for $-2 < \zeta < 0$ (unstable)}\\
  1 + \beta \zeta  & \text{for $0 \le \zeta < 1$ (stable)}
\end{cases}
\end{equation}

\begin{equation}
  \label{eq:busdy2}
\phi_H = 
\begin{cases}
  Pr_t ( 1 - \gamma_2 \zeta )^{-1/2}  &  \text{for $-2 < \zeta < 0$ (unstable)}\\
  Pr_t + \beta \zeta  & \text{for $0 \le \zeta < 1$ (stable)}
\end{cases}
\end{equation}
where $Pr_t$ is the turbulent Prandtl number = $K_M/K_H$ for neutral stability ($\zeta=0$).

As Bretherton (UW547 Lecture 6) explains, there is some leeway in the fit coefficients.
Roland using the Kansas (Businger et al., 1971) values of $Pr_t=0.74$, $\beta=4.7$ $\gamma_1 = 15$, $\gamma_2=9$,
Dyer, 1974 recommends $Pr_t=1$, $\beta=5$ $\gamma_1 = 16$, $\gamma_2=16$ (see Garrett Appendix 4).
If we adopt the Dyer values and plug them into (\ref{eq:ri-mo}) we get

\begin{equation}
  \label{eq:busdy3}
\zeta = 
\begin{cases}
  Ri  &  \text{for $-2 < \zeta < 0$ (unstable)}\\
  \frac{Ri }{1 - 5 Ri}  & \text{for $0 \le \zeta < 0.2$ (stable)}
\end{cases}
\end{equation}
 
\section{Integrating the surface layer winds}
\label{sec:integr-surf-layer}

\subsection{Stable layer}
\label{sec:stable-layer}



To get dimensional winds from the non-dimensional gradients, recall that for the neutral
surface layer

\begin{equation}
  \label{eq:neutral}
  \frac{ \partial \overline{ u}}{\partial z}  = \frac{ u_*}{kz} 
\end{equation}
(Stull 9.7.1c on p.~378), and in the non-neutral surface layer we can rearrange~(\ref{eq:surface})
to get:

\begin{equation}
  \label{eq:non-neutral}
\frac{ k}{u_*} \int_{z_0 }^{z}\! d \overline{ u}
= \int_{z_0 }^{z}\!\phi_M(\zeta) \frac{ d z^\prime}{z} = \int_{z_0 }^{z}\!\phi_M(\zeta) \frac{ d \zeta^\prime}{\zeta}
\end{equation}
since $\zeta = z/L$.  The lhs is a perfect differential, and for the rhs use the
trick of adding and subtracting a factor of $\frac{d\zeta^\prime }{\zeta^\prime}$ to get

\begin{equation}
  \label{eq:integbus}
  \frac{ k u}{u_*} = \int_{z_0 }^{z}\! \frac{ d \zeta^\prime}{\zeta}- \int_{z_0 }^{z}\!(1 - \phi_M(\zeta)) \frac{ d \zeta^\prime}{\zeta}
  = \ln \left ( \frac{z }{z_0} \right ) - \Psi_M(\zeta)
\end{equation}
which is Stull 9.7.5g on p.~385 (note the sign change in front of $\Psi_M$).

Simlarly, the dimensional $\theta$ profile is:

\begin{equation}
  \label{eq:integbus2}
  \frac{ k \theta}{\theta_*} =  \ln \left ( \frac{z }{z_{T0}} \right ) - \Psi_H(\zeta)
\end{equation}

If we take the Dyer fit for the stable layer $\phi_M$ of $\phi_M = 1 + 5 \zeta$ we
can integrate (\ref{eq:integbus}) to get 

\begin{equation}
  \label{eq:integbus3}
\Psi_M(\zeta) = \int_{0 }^{\zeta}\!(1 - \phi_M(\zeta)) \frac{ d \zeta^\prime}{\zeta^\prime}
  = \int_{0 }^{\zeta}\!(1 - (1 + 5 \zeta^\prime )) \frac{ d \zeta^\prime}{\zeta^\prime}
  = \int_{0 }^{\zeta}\!(- 5 )) d \zeta^\prime = -5 \zeta
\end{equation}
for $0 < \zeta < 1$.  This means that we have for the stable layer that

\begin{equation}
  \label{eq:stable-final}
  \frac{ u}{u_*} = k^{-1} \left [ \ln \left ( \frac{ z}{z_0} \right ) + \frac{ 5 z}{L}  \right ]
\end{equation}
(Stull 9.7.5g on p.~385), and $u$ increases more rapidily with height away from the surface, as
shown in Figure 9.5 on  p.~377.

\subsection{Unstable layer}
\label{sec:unstable-layer}

For unstable conditions we need to
integrate $\phi_M(\zeta)=(1-16\zeta)^{-1/4}$ and
$\phi_H(\zeta)=(1-16\zeta)^{-1/2}$.  We'll do $\phi_M$ here
and leave $\phi_H$ as an exercise.  Change variables
to $\phi_M^{-1}=(1 - 16\zeta)^{1/4} = x$.  Then
$\zeta=\frac{ 1-x^4}{4}$ and $d\zeta = - x^3 dx$.  Then just as before we
need to integrate:

\begin{equation}
  \label{eq:unstable1}
  \Psi_M = \int_{0 }^{\zeta}\!(1 - \phi_M(\zeta)) \frac{ d \zeta^\prime}{\zeta^\prime}
= \int_{ 0}^{\zeta}\! (1 - x^{-1}) \frac{ -4 x^3 dx}{(1 - x^4)} 
\end{equation}
We can tackle (\ref{eq:unstable1}) using partial fractions and get,
for the unstable regime $-2 < \zeta < 0$:

\begin{align}
  \Psi_M &= \ln \left [ \left ( \frac{ 1 + x^2}{2} \right ) 
\left ( \frac{ 1 + x}{2} \right )^2 \right ] - 2\tan^{-1} x + \pi/2\\
  \Psi_H & = 2 \ln \left ( \frac{ 1 + x^2}{2} \right )
\end{align}

\subsection{Exchange coefficients}
\label{sec:exch-coeff}

Now use all this to get an expression for the exchange coefficients $C_D$ and
$C_H$:

\begin{gather}
\text{The definition of $C_D$: } u_*^2 = C_D U^2\\
\text{MO expression for U}: \frac{ k U}{u_*} = \ln \left ( \frac{ z}{z_0} \right ) - \Psi_M(\zeta)
\end{gather}
put these together to get:

\begin{equation}
  \label{eq:drag}
  C_D = \frac{ u_*^2}{U^2} = \frac{ k^2}{ \ln(z/z_0) - \Psi_M(\zeta)} 
\end{equation}


And for energy or moisture:

\begin{gather}
\text{The definition of $C_H$: } \left (\overline{ w^\prime \theta_v^\prime } \right )  = 
         C_H U (\theta_{v0} - \theta_v) \label{eq:ch}\\
\text{MO expressure for $\theta_v$}: \frac{k(\theta_{v0} - \theta_v)}{\theta_{v*}} = 
\ln \left ( \frac{ z}{z_T} \right ) - \Psi_H(\zeta)
\end{gather}
Put the MO relationships for $\theta$ and $U$ into \eqref{eq:ch} to get:

\begin{equation}
  \label{eq:chfinal}
u_* \theta_{v*} = C_H   \frac{\left [\ln \left ( \frac{ z}{z_T} \right ) - \Psi_H(\zeta)\right ] }{k} 
\frac{ \left [  \ln \left ( \frac{ z}{z_0} \right ) - \Psi_M(\zeta)\right ] }{k} u_* \theta_* 
\end{equation}
or neglectingthe differnce between $\theta_{v*}$ and $ \theta_* $:

\begin{equation}
  \label{eq:chfinal}
 C_H = k^2   \left [\ln \left ( \frac{ z}{z_T} \right ) - \Psi_H(\zeta)\right ]^{-1} 
\left [  \ln \left ( \frac{ z}{z_0} \right ) - \Psi_M(\zeta)\right ]^{-1} 
\end{equation}

So you can find the fluxes of momentum and energy: $\overline{ w^\prime \theta_v^\prime }$
and $\overline{u^\prime w^\prime}$ at height $z$ if you know
the $L$.

\end{document}
  








%%% Local Variables:
%%% mode: latex
%%% TeX-master: t
%%% End:
