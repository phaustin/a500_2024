\documentclass[12pt]{article}
\usepackage{geometry,fancyhdr,xr,hyperref,ifpdf,amsmath,rcs,shortcuts,amsfonts}
\usepackage{lastpage,longtable,color,paunits,amsmath,smallsec}
%\usepackage{showlabels}
\geometry{letterpaper,top=50pt,hmargin={20mm,20mm},headheight=15pt} 


\pagestyle{fancy} 

\RCS $Revision: 1.1 $

\RCS $Date: 2007/11/23 17:26:55 $
\fancypagestyle{first}{
\lhead{A500}
\chead{Mixed layer equations I}
\rhead{page~\thepage/\pageref{LastPage}}
\lfoot{} 
\cfoot{} 
\rfoot{}
}

\ifpdf
    \usepackage[pdftex]{graphicx} 
    \usepackage{hyperref}
    \pdfcompresslevel=0
    \DeclareGraphicsExtensions{.pdf,.jpg,.mps,.png}
\else
    \usepackage{hyperref}
    \usepackage[dvips]{graphicx}
    \DeclareGraphicsRule{.eps.gz}{eps}{.eps.bb}{`gzip -d #1}
    \DeclareGraphicsExtensions{.eps,.eps.gz}
\fi

\newcommand{\vect}[1]{\mathbf{{#1}}}

\begin{document}

\pagestyle{first}

\begin{center}
Atsc500: Mixed layer equations\\
\end{center}
\section{Review}
\label{sec:sum9}

From 
\href{https://www.dropbox.com/scl/fi/e6cq3sodf0fq2rwuynsfm/boussinesq.pdf?rlkey=8josihsmcn1fskhl6mvnlu4eh&dl=0}{the Boussinesq and anelastic notes} we've got the following:


\begin{itemize}
\item Boussinesq continuity equation:

\begin{gather}
  \label{eq:bous2}
\nabla \vect{v} =0
\end{gather}


\item Anelastic continuity equation:

\begin{equation}
  \label{eq:anel2}
  \nabla ( \rho_0(z) \vect{v} ) = 0
\end{equation}


\item We also found two thermodynamic relationships between
temperature, potential temperature, density and pressure in the anelastic
approximation, given for example by 
\href{https://journals.ametsoc.org/doi/abs/10.1175/1520-0469%281995%29052%3C2302%3APVCHAA%3E2.0.CO%3B2}{Bannon 1995} eq 2.17:

\begin{gather}
  \frac{ \theta}{\theta_0} \sim \frac{T }{T_0} 
\end{gather}

and the anelastic equation of state:

\begin{gather}
  \frac{p }{p_0}  \sim \frac{ \rho}{\rho_0}  + \frac{T }{T_0} 
\end{gather}


\item and for the Boussinesq approximation, these reduce to 
\href{/Users/phil/Dropbox/phil_files/a500/docs/bannon95a.pdf}{Bannon 1995} (5.20)
\begin{gather}
  \frac{ \theta}{\theta_*} \sim \frac{T }{T_*} \\
\theta_* \sim T_* \\
\theta \sim T
\end{gather}
where $T_*$ and $\theta_*$ are numbers, not functions of height.  Also remember
that $T$, etc. are not the total temperature $T_T$, but rather 
the difference between the temperature and the base state, $T_T - T_0$.

The equation of state for the Boussinesq approximation is Bannon (5.20e):
\begin{gather}
  \frac{ \rho}{\rho_*}  \sim  - \frac{T }{T_*} 
\end{gather}


\item You should  also show that conservation of entropy can be written
as Bannon (2.3c)

       \begin{equation}
        \label{eq:base}
         \frac{d \theta }{dt}  + w \frac{ d \theta_0}{dz}  = \frac{\theta_0 Q}{\rho_0 c_p T_0} 
       \end{equation}

which for the Boussinesq approximation simplifies to 

       \begin{equation}
         \frac{d \theta }{dt}  + w \frac{ d \theta_0}{dz}  = \frac{ Q}{\rho_* c_p } 
       \end{equation}
with $Q$ representing the heating rate.

To simplify things we can make the base state $\theta_0$ independent of height,
so $w \frac{d \theta_0}{dz} =0$ and we can specify no heating for the base
state, so $Q_0=0$ (we've already assumed this in getting (\ref{eq:base}).

\end{itemize}

\section{Mixed layer equations}
\label{sec:mixed-layer-equat}

Now how do we go from these to something useful for the case in which
turbulence is producing constant profiles of $\theta$ and $q_T$ (the
total specific humidity) below an inversion at height $h=z_i$?  
(See Stull Figure3.7, p. 104 for model and observations of mixed
layer growth).

Stull
presents equations for this in Chapter 7, p. 271.  In this section, I'm using the approach of
\href{http://ezproxy.library.ubc.ca/login?url=http://link.springer.com/10.1007/s00162-006-0032-z}{Stevens (2005)} to show how Stull 7.4.3 arises from
the continuity equation and conservation of energy via
the ``mixed layer equations''. This is a set of
three equations in three unknowns:
\begin{gather}
  \text{mean temperature: } \frac{d \hat{\theta} }{dt} =(1 + k) F_0/(h c_p\rho_*)\label{eq:meantheta}\\
\text{inversion height: } \frac{dh }{dt} = ( k F_0)/(\rho_* c_p \Delta \theta) + w_h \label{eq:hrise}\\
\text{inversion jump: } \frac{d \Delta \theta }{dt} = \left ( \frac{dh }{dt}  - w_h \right ) \Gamma - \frac{d \hat{\theta} }{dt}\label{eq:jump}
\end{gather}
where all the symbols will be defined below. (Stevens writes $h$ for 
Stull's  inversion top height $z_i^+$)

\subsection{Definitions}

Start with the definition of the mixed layer average.  Based on figure
3.7 I know that in a well mixed dry boundary layer the potential
temperature below the inversion can be represented as a single number:
the vertical average.  I'll use the ``hat'' ($\widehat{\ }$) symbol
for this:

\begin{equation}
  \label{eq:vertavg}
  \widehat{\theta} = \frac{1}{h} \int_0^h \theta dz
\end{equation}


a boundary layer of height $h$ which varies with time but is assumed
flat (i.e. $\nabla_H\,h = \partial_x h + \partial_y h = 0$. Stevens (2005)
shows how to relax this assumption and consider a sloping inversion).  Some
definitions: $\vect{u_H}=\mbox{horizontal velocity vector (u,v)}$,
subscript $0$ denotes a surface quantity, subscript $h$ denotes an
inversion-top,  property, and the vertical average is given by:

\begin{equation}
  \label{eq:vertavgphi}
  \widehat{\phi} = \frac{1}{h} \int_0^h \phi dz
\end{equation}


\textbf{What to watch out for}:  As explained in lecture, there is an important distinction
between $h$, which is the top of the inversion, and $z_i$, which
is the base.  Only cumulus cloud tops can penetrate into $h$, whereas
$z_i$ is in a fully turbulent environment at the top of the mixed
layer.  The distance $\epsilon=h - z_i$ is perhaps 10 m under conditions
of strong subsidence.



\subsection{Continuity equation}
\label{sec:continuity}

Assume the Boussinesq approximation:

\begin{equation}
  \label{eq:incomp}
\nabla_H \cdot \vect{u_H} =  -\partial_z w
\end{equation}

Integrate (\ref{eq:incomp}) over the boundary layer depth, from the surface ($z=0$), to
just on top of the inversion jump, where turbulent fluxes vanish ($z_h = h$).  As noted above,
we'll assume
that the inversion has a thickness $\epsilon$, so that $(h - z_i) = \epsilon$, where $z_i$ is the
height of the inversion base, inside the well-mixed layer.  We'll look at the inversion jump
in more detail in Section~\ref{sec:anoth-view-entr} below.

\begin{equation}
  \label{eq:intcomp}
 \int_0^h \nabla_H \cdot \vect{u_H} \, \, dz = \int_0^h -\partial_z w \, dz
\end{equation}

Using Liebniz' rule and (\ref{eq:vertavg}) on the lhs of (\ref{eq:intcomp}) gives:

\begin{equation}
  \label{eq:complhs}
  \int_0^h \nabla_H \cdot \vect{u_H}   dz = \nabla_H \cdot \int_0^h \vect{u_H} \,\,dz
               - \vect{u^h_H} \nabla_H\, h = \nabla_H \, h \widehat{\vect{u_H}}
= h \nabla_H \widehat{\vect{u_H}}
\end{equation}
since $\nabla_H h = 0$

Using (\ref{eq:complhs}) and doing the integration of the rhs of (\ref{eq:intcomp}):

\begin{equation}
  \label{eq:finalcont}
  h \nabla_H \widehat{\vect{u_H}} = -w_h
\end{equation}
How would you illustrate (\ref{eq:finalcont}) with a simple cartoon? Note that if there
is divergence in the column, $w_h$ is negative at the top of the inversion.  In a region of subtropical
high pressure the value of $w_h$ will be a few cm/s.

\section{First law}
\label{sec:first-law}

Now back to our energy equation where in the Boussinesq approximation we
can use either $\theta$ or the static energy residual 
$\phi = \phi_T - \phi_0 = c_p (T_T - T_0 ) + g (z_T - z_0)$:

\begin{gather}
  \label{eq:first}
  \frac{d \theta}{dt} = \frac{q }{\rho_* c_p}\\
\intertext{or}
\frac{d \phi}{dt} = q
\end{gather}
where $q$ is the heating rate due to radiation and evaporation of
precipitation falling into the layer.  I'll switch to the static
energy below, because it works for the both the anelastic
and Boussinesq approximations and it's easier to type.
Multiplying both sides by
$\rho_0$ and converting to flux form as usual gives:

\begin{equation}
  \label{eq:firsts}
  \frac{d \rho_0 \phi}{dt} = \rho_0\, q
\end{equation}


We'll change to concentration units for the static energy and write $s=\rho \phi$.
It's (finally) time to to a Reynold's decomposition (\ref{eq:firsts}),
which looks like:

\begin{equation}
  \label{eq:reynolds}
  \underbrace{\partial_t \overline{s}}_1 + \underbrace{\overline{u_H} \cdot \nabla_H \overline{s}}_2
+ \underbrace{\overline{w} \partial_z \overline{s}}_3 = -\underbrace{\nabla_H \overline{\vect{u_H^\prime} s^\prime}}_4
- \underbrace{\partial_z \overline{w^\prime s^\prime}}_5 + \underbrace{\overline{\rho}\,\overline{q}}_6
\end{equation}
\textbf{Below I'll drop the over bar on all ensemble mean quantities except for the 
correlations.} Note that we are now dealing with two different averages: Reynolds and vertical.
We want to integrate (\ref{eq:reynolds}) vertically to get the mixed layer relations.  Start with the three
terms on the lhs and use Leibniz' rule:


\subsection{lhs}

\noindent
\textit{\textbf{Term 1 of (\ref{eq:reynolds})}}: $\partial_t \overline{s}$

\begin{equation}
  \label{eq:term1}
  \int_0^h \partial_t s dz = \partial_t \int_0^h s dz - s_h \partial_t h
\end{equation}
and using (\ref{eq:vertavg}) this becomes:

\begin{equation}
  \label{eq:term1b}
  \int_0^h \partial_t s dz = \partial_t ( h \hat{s} ) - s_h \partial_t h
= h \, \partial_t \hat{s} + \hat{s} \partial_t\,h - s_h \partial_t h
= h \, \partial_t \hat{s} - (s_h -  \hat{s}) \, \partial_t\,h 
\end{equation}

\noindent
\textbf{\textit{Second term of (\ref{eq:reynolds}):}} $\overline{u_H} \cdot \nabla_H \overline{s}$

\begin{equation}
  \label{eq:second}
  \int_0^h \vect{u_H} \cdot \nabla_H s \, dz =
\int_0^h \, \nabla_H \cdot ( \vect{u_H} \, s ) \, dz -  
\int_0^h\, s \nabla_H \cdot \vect{u_H} dz
\end{equation}

Again using Leibniz' rule (\ref{eq:second}) becomes:

\begin{subequations}
\begin{eqnarray}
  \label{eq:secondb}
  \int_0^h \vect{u_H} \cdot \nabla_H s \, dz & = &
\nabla_H \cdot \int_0^h\,\vect{u_H} \,s   \, dz - \vect{u_{H\, +}}\, s_h \nabla_H h
- \int_0^h\, s\, \nabla_H \cdot \vect{u_H} \, dz \mbox{ or: } \\
\int_0^h \vect{u_H} \cdot \nabla_H s \, dz &=& \nabla_H \cdot ( h \, \widehat{\vect{u_H}} \hat{s}) - \int_0^h\, s\, \nabla_H \cdot \vect{u_H} \, dz \label{eq:secondc}
\end{eqnarray}
\end{subequations}
Where we've assumed that $\widehat{\vect{u_H}\,s}=\widehat{\vect{u_H}}\,\hat{s} $ (This is called the \textit{allotropic}
assumption)


Finally look at the \textbf{\textit{third term of (\ref{eq:reynolds}):}} $\overline{w} \partial_z \overline{s}$

\begin{equation}
  \label{eq:third}
  \int_0^h \,  w \, \partial_z s \, dz = \int_0^h\,\partial_z ( w\,s) \, dz
- \int_0^h\,s \partial_z \, w \, dz = 
\int_0^h\,\partial_z ( w\,s) \, dz
+ \int_0^h\,s \nabla_H \, \vect{u_H} dz
\end{equation}
where I've used the continuity equation~(\ref{eq:incomp}).  Note that the last
term of (\ref{eq:third})  is going to cancel the last term of (\ref{eq:secondc}) when
I combine all the lhs terms.

Notice that since $w_0 = 0$ at the surface, we have:

\begin{equation}
  \label{eq:iter1}
  \int_0^h\,\partial_z ( w\,s) \, dz = w_h\,s_h
\end{equation}

\noindent
\textit{\textbf{Finish lhs}}: Now collect all of these terms for a complete lhs for the vertically integrated
form of (\ref{eq:reynolds}):

\begin{equation}
  \label{eq:reynlhs}
lhs = h \partial_t\, \hat{s} + h \widehat{\vect{u_H}} \cdot \nabla_H \hat{s} - (s_h - \hat{s} ) \partial_t\,h + h \hat{s} \, \nabla_H \cdot \widehat{\vect{u_H}} - w_h\,s_h  
\end{equation}

If we define a new \textit{horizontal only} version of the total derivative
following the flow:



\begin{equation}
  \label{eq:hortotal}
  \frac{d \hat{s}}{dt} = \partial_t\, \hat{s} + \widehat{\vect{u_H}} \cdot \nabla_H \hat{s}
\end{equation}
and use the vertically integrated continuity equation (\ref{eq:finalcont}) for $h\,\nabla_H \cdot \widehat{\vect{u_H}}$ then
(\ref{eq:reynlhs}) can be rewritten as:

\begin{equation}
  \label{eq:reyn2}
lhs =  h \,\frac{d \hat{s}}{dt} - (s_h - \hat{s} )
\left (\frac{dh}{dt} - w_h \right )
\end{equation}
Be sure you can explain, in words with the help of a picture, what
(\ref{eq:reyn2}) represents.


\subsection{rhs of (\ref{eq:reynolds}):}
\label{sec:rhs}

OK, we're ready for the rhs of (\ref{eq:reynolds}).  First, make two assumptions.

1) Assume that the horizontal turbulent fluxes are horizontally uniform, so
we can write:

\begin{equation}
  \label{eq:lose1}
  \nabla_H \overline{\vect{u_H^\prime} s^\prime} = 0
\end{equation}

2) Write the heating rate as the divergence of a flux $R$:

\begin{equation}
  \label{eq:heating}
  \overline{\rho}\, \overline{q} = -\partial_z \ R
\end{equation}

With these two simplifications I lose term (4) and I only have to integrate perfect differentials
with height, so the vertically integrated rhs of (\ref{eq:reynolds}) becomes:


\begin{equation}
  \label{eq:reynoldsint}
rhs = \int_0^h \,- \partial_z \overline{w^\prime s^\prime}  - \partial_z \ R \,dz
= - \left ( \overline{w^\prime s^\prime}_h - \overline{w^\prime s^\prime}_0 \right ) - ( R_h  - R_0 )
\end{equation}
As before, what is this saying, in words?

\section{Full vertically averaged equation}
\label{sec:fullint}

Combining both sides, we now have:

\begin{equation}
  \label{eq:combine}
  h \,\frac{d \hat{s}}{dt} - (s_h - \hat{s} )
\left (\frac{dh}{dt} - w_h \right ) = - \left ( \overline{w^\prime s^\prime}_h - \overline{w^\prime s^\prime}_0 \right ) - ( R_h  - R_0 )
\end{equation}

Let's make two more assumptions:


1) The transport at the top of the inversion inversion represented by the 
$\overline{w^\prime s^\prime}_h$ term in (\ref{eq:combine}) is assumed to be due to
small cumulus clouds and can be modelled as a positive velocity $M$ moving material
upward through the inversion (remember from the beginning of Section \ref{sec:continuity} that
turbulent fluctuations are damped at $h=z_+$, so $\overline{w^\prime s^\prime}_h$ can't be 
due to normal turbulent Reynolds fluctuations):
\begin{equation}
  \label{eq:cumulus}
  \overline{w^\prime s^\prime}_h = -(s_h - \hat{s} )\,M
\end{equation}
We'll be setting $M$, the cloud mass flux, to zero for now (dry boundary layer)
and coming back to it later in the term.


2) We'll see in chapter 7 that the surface flux $\overline{w^\prime s^\prime}_0$ can be modelled 
using a bulk transport coefficient as a velocity V depends on the
and the mean horizontal  wind:

\begin{equation}
  \label{eq:surface}
  \overline{w^\prime s^\prime}_0 = -(\hat{s} - s_0 )\,V = F_0
\end{equation}
where $0$ subscript denotes the surface (z=0)

With these two assumptions the final form of (\ref{eq:combine}) is:

\begin{equation}
  \label{eq:combine1}
  h \,\frac{d \hat{s}}{dt} - (s_h - \hat{s} )
\left (\frac{dh}{dt} - w_h - M \right ) =  (\hat{s} - s_0 )\,V - ( R_h  - R_0 )
\end{equation}

We \textit{define} the entrainment rate, $w_e$ as:

\begin{equation}
  \label{eq:entrain}
  w_e = \frac{dh}{dt} - w_h - M
\end{equation}

Using (\ref{eq:entrain}), (\ref{eq:combine1}) becomes:


\begin{equation}
  \label{eq:combine2}
 \frac{d \hat{s}}{dt} = \frac{ (s_h - \hat{s} )w_e + F_0 - ( R_h  - R_0 )}{h}
\end{equation}

Be sure you can sketch a physical rationale for each of the terms in (\ref{eq:combine2}).
In the next section I'll relate the entrainment velocity $w_e$ to the flux at the
\textit{base} of the the inversion, $\overline{w^\prime s^\prime}_{z_i}$  For now, note the resemblance between
(\ref{eq:combine2}) and \ref{eq:meantheta}.  If we write $(s_h - \hat{s} )w_e = k F_0$,
neglect the radiative/precip fluxes $F$, and recognize that under the Boussinesq
approximation $\rho* c_p \theta = s$,  we have derived (\ref{eq:meantheta}).



\section{Another view of entrainment: mass conservation across the inversion}
\label{sec:anoth-view-entr}

We can get a slightly different perspective on the top of the mixed
layer by looking at mass continuity across a thin layer
stretching from the top of the inversion layer $h=z_h$ down to the inversion base at $z_i$.
As we noted in assumption 1) in Section~\ref{sec:fullint}, there is no
turbulence at $z_h$ except for the occasional cloud pumping air into
the layer above the inversion.  But what about the flux at the top of the mixed layer
(base of the inversion),
$(\overline{w^\prime s^\prime})_{z_i}$?.  We want to get the following entrainment equation

\begin{equation}
  \label{eq:ps5}
  - (s_h - \hat{s}) \cdot E = \left ( \overline{w^\prime s^\prime}_i \right ) \overline{\rho}(z_i)
\end{equation}

To see how to do this starting from energy conservation, we'll first look at the continuity
equation to get $E$, then use conservation of energy in the final section to get (\ref{eq:ps5}).

\begin{equation}
  \label{eq:cont}
  \partial_t \rho + \partial_z (\rho w) = 0
\end{equation}

Integrate (\ref{eq:cont}) from the mixed layer top to above the jump:

\begin{equation}
  \label{eq:contint}
  \int_{z_i}^{h} \! \partial_t \rho  \,dz = - \int_{z_i}^{h}\! \partial_z (\rho w) \,dz
\end{equation}

Use Leibniz' rule as usual on the lhs of (\ref{eq:contint}):

\begin{equation}
  \label{eq:contleib}
 \partial_t \int_{z_i}^{h} \!  \rho  \,dz = \int_{z_i}^{h} \! \partial_t \rho  \,dz + \rho(h) \partial_t h
 - \rho(z_i) \partial_t z_i
\end{equation}

Remembering that $h - z_i = \epsilon$, we can define an inversion average as:

\begin{equation}
  \label{eq:invavg}
  \widetilde{\rho} = \int_{z_i}^{h} \!  \rho  \,dz / \epsilon
\end{equation}
Using (\ref{eq:invavg}) and (\ref{eq:contleib}) we can rewrite (\ref{eq:contint}) as:

\begin{equation}
  \label{eq:contint2}
  \epsilon \widetilde{\rho} - ( \rho(h) - \rho(z_i)) \partial_t h = - \left [ (\rho\, w)_h - (\rho\, w)_{z_i} \right ]
\end{equation}
where we've assumed that the inversion thickness $\epsilon$ stays constant with time, so that $\partial_t h = \partial_t z_i$.

If we let
the inversion thickness $\epsilon \rightarrow 0 $, we can write (\ref{eq:contint2}) as:

\begin{equation}
  \label{eq:contint3}
 \rho(h)(\partial_t h - w(h)) = \rho(z_i) (\partial_t h  - w(z_i) ) 
\end{equation}

Equation (\ref{eq:contint3}) defines the entrainment rate:

\begin{gather}
  \label{eq:erate}
  E = \rho(z_i) (\partial_t h  - w(z_i) )\\
\intertext{or to put it in the symbols of equation (\ref{eq:hrise})}
E \sim \rho_* \left (\frac{dh }{dt}  - w_h \right )
\end{gather}

and the entrainment velocity is defined as:

\begin{equation}
  \label{eq:evel}
  w_e = E / \rho(z_i)
\end{equation}

Physically, $E$ represents the mass flux across the inversion base, due to large scale subsidence and
the action of turbulence bringing inversion air into the mixed layer.


\textbf{\textit{Question: How would add cumulus transport, M, to this?}}  (compare (\ref{eq:erate}) with (\ref{eq:entrain})).

\section{What about the turbulence?: energy conservation across the inversion}
\label{sec:what-about-turb}

Note that I haven't done any Reynolds averaging to get (\ref{eq:erate}) (how come?).  I've got $E$, the
entrainment rate, but I'm still not all the way to (\ref{eq:ps5}), which I'll repeat here:


  \begin{equation}
    \label{eq:hd0}
  - (s_h - \hat{s}) \cdot E  =  - \Delta s \cdot E = \left ( \overline{w^\prime s^\prime} \right )_{z_i} \overline{\rho}(z_i)
  \end{equation}

Look at the first law in flux form, neglecting the horizontal advection (why is this ok?) and
\textbf{\textit{assuming no radiation}} (debatable) or precipitation in the thin inversion layer.

\begin{equation}
  \label{eq:reynoldsb}
  \partial_t (\rho\,\phi) + \partial_z \, (w \,\rho\,\phi) =
- \partial_z \overline{\rho\, w^\prime \phi^\prime} 
\end{equation}
Where  as before I've dropped the overbar on all mean quantities except for the 
correlations.  


Now as before integrate (\ref{eq:reynoldsb}) across the inversion:

\begin{equation}
  \label{eq:reynoldsInt}
\int_{z_i}^{h} \! 
  \partial_t (\,\rho\,\phi )\, dz  + 
 \int_{z_i}^{h} \! \partial_z \, (\rho\,w \,\phi) \, dz  =
 - \int_{z_i}^{h} \partial_z \overline{\rho\, w^\prime \phi^\prime} dz
\end{equation}


\noindent
\textit{\textbf{Term 1 of (\ref{eq:reynoldsInt})}}:

\begin{equation}
  \label{eq:term1c}
\int_{z_i}^{h} \! 
  \partial_t (\,\rho\,\phi )\, dz = \partial_t \int_{z_i}^{h} \! 
   (\,\rho\,\phi )\, dz - (\,\rho\,\phi )_+ \partial_t h  +  (\,\rho\,\phi )_i \partial_t z_i
\end{equation}
As before, if the inversion thickness $\epsilon$ is constant with time then $\partial_t h = \partial_t z_i$.
Using (\ref{eq:invavg}) we can then write (\ref{eq:term1c}) as:

\begin{equation}
  \label{eq:term1d}
\int_{z_i}^{h} \! 
  \partial_t (\,\rho\,\phi )\, dz = \partial_t \,\left (- \epsilon \,\widetilde{\rho \phi} \right )
     + ((\rho \phi)_h - (\rho \phi)_{z_i}) \partial_t h = ((\rho \phi)_h - (\rho \phi)_{z_i}) \partial_t h
\end{equation}
where we've taken $\lim{{\epsilon \rightarrow 0}}$



\noindent
\textbf{\textit{Second term of (\ref{eq:reynoldsInt}):}}

The second term integrates to:

\begin{equation}
  \label{eq:secondd}
(w\,\rho\,\phi)_h - (w\,\rho\,\phi)_{z_i} = \overline{w} (s_h - s_{z_i}) = \overline{w} (s_h - \widetilde{s})
\end{equation}
where I've used the fact that the large scale velocity $w$ is constant across the inversion and
that since the layer is well-mixed, $s_{z_i} = \widetilde{s}$.

\noindent
\textit{\textbf{rhs}}

If we assume no turbulence in the inversion, then $\overline{\rho\, w^\prime \phi^\prime}_h$=0 and

\begin{equation}
  \label{eq:rhs}
   - \int_{z_i}^{h} \partial_z \overline{\rho\, w^\prime \phi^\prime} dz = - \left ( \overline{\rho\, w^\prime \phi^\prime}_h -
\overline{\rho\, w^\prime \phi^\prime}_{z_i} \right ) = \overline{\rho\, w^\prime \phi^\prime}_{z_i} = F_i
\end{equation}

\section{Final version}

Combining the terms from (\ref{eq:term1d}), (\ref{eq:secondd}) and (\ref{eq:rhs}) and moving everything to the rhs yields:

\begin{equation}
  \label{eq:final}
(s_h - s_{z_i}) (\partial_t h - \overline{w}_{z_i}) =  F_i =  \overline{\rho\, w^\prime \phi^\prime}_{z_i}
\end{equation}

In words, equation (\ref{eq:final}) states that the flux $F_i$ at the inversion base
changes the properties of the air coming into the layer from above from $s_h$ to $s_{z_i}$.  Since
the inversion air is usually warmer and dryer than air in the mixed layer, this means that $F_i$ acts
to cool and moisten the entrained inversion air.  This, is our second mixed layer equation (\ref{eq:hrise})
in disguise.

\section{Recap}
\label{sec:recap}

So far we've derived two of the three equations.  To review:
Equation \eqref{eq:meantheta} is the same as Equation
\eqref{eq:combine2}

\begin{gather*}
  \text{mean temperature: } \frac{d \hat{\theta} }{dt} =(1 + k) F_0/(h c_p\rho_*)\\
\text{mean static energy: } \frac{d \hat{s}}{dt} = \frac{ (s_h - \hat{s} )w_e + F_0 - ( R_h  - R_0 )}{h}
\end{gather*}
because we've shown that $k F_0 = F_{z_i}= w_e \Delta s$ and we're assuming
we can neglect the radiative and precipitation fluxes $R_x$ at $h$ and
the surface and in the Boussinesq approximation $\hat{\theta}$ =
$\hat{T} = \hat{s}/(\rho_* c_p)$

Also Equation \eqref{eq:hrise} is equivalent to \eqref{eq:final}:

\begin{gather*}
\text{inversion height: } \frac{dh }{dt} = ( k F_0)/(\rho_* c_p \Delta \theta) + w_h \label{eq:hriseb}\\
(s_h - s_{z_i}) (\partial_t h - \overline{w}_{z_i}) =  F_i =  \overline{\rho\, w^\prime \phi^\prime}_{z_i}
\end{gather*}
provided we neglect the difference between $w(h)$ and $w(z_i)$
predicted by (\ref{eq:contint3}).



\end{document}

%%% Local Variables:
%%% mode: latex
%%% TeX-master: t
%%% End:
