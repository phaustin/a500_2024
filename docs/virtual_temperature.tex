\documentclass[12pt]{article}
\usepackage{geometry,fancyhdr,xr,hyperref,ifpdf,amsmath,rcs,indentfirst}
\usepackage{lastpage,longtable,Ventry,url,paunits,shortcuts,smallsec,color,tightlist,float}
\geometry{letterpaper,top=50pt,hmargin={20mm,20mm},headheight=15pt}
\usepackage[stable]{footmisc}

\pagestyle{fancy} 

\RCS $Revision: 1.5 $
\RCS $Date: 2002/01/09 03:50:54 $

\fancypagestyle{first}{
\chead{Thermo review: Virtual temperature}
\lhead{2019/Sep/10}
\rhead{page~\thepage/\pageref{LastPage}}
\lfoot{} 
\cfoot{} 
\rfoot{}
}

\ifpdf
    \usepackage[pdftex]{graphicx} 
    \usepackage{hyperref}
    \pdfcompresslevel=0
    \DeclareGraphicsExtensions{.pdf,.jpg,.mps,.png}
\else
    \usepackage{hyperref}
    \usepackage[dvips]{graphicx}
    \DeclareGraphicsRule{.eps.gz}{eps}{.eps.bb}{`gzip -d #1}
    \DeclareGraphicsExtensions{.eps,.eps.gz}
\fi

\begin{document}
\pagestyle{first}


\section{Virtual temperature}
\label{sec:virtual-temperature}

A nice example of the use of Taylor's series is Stull (1.5.1a) on p.~7:

\begin{equation*}
  \theta_v = \theta ( 1 + 0.61 r_{sat} - r_l )
\end{equation*}
where $\theta$ called the \textit{potential temperature} is acutally a measure
of the entropy of dry air (to be derived later), $r_{sat}$ is the saturation
mixing ratio for water vapor (kg water/kg dry air) and $r_l$ is the
mixing ratio for cloud droplets (kg water/kg dry air).

We'll go into all of this in more detail, but it's worth looking at how
the factor 0.61 comes about in equations for density like 
\href{https://www.eoas.ubc.ca/books/Practical_Meteorology/prmet102/Ch03-thermo-v102b.pdf}%
{Stull PM Chapter 3 eq. 3.15)}
Virtual temperature provides a succinct way to describe the density of a mixture of
dry air, water vapor, and hydrometeors like cloud droplets, raindrops, snow and ice.
Specifically, write this density as:

  \begin{equation}
    \label{eq:dalton}
    \rho = \rho_d + \frac{e}{R_v\, T} + \rho_l + \rho_r + \rho_i
  \end{equation}
where $\rho_d$ is the density of dry air, $e$ is the vapour pressure, $R_v$ is the
gas constant for water vapor (461 \jkgk), $T$ is the temperature and
$\rho_l,\  \rho_r,\ \rho_i$ are the densities of cloud droplets, rain drops and ice
crystals.  (When we begin comparing densities to find the \textit{buoyancy} we'll
need the additional assumption that the droplets, drops and crystals are all
falling at constant velocity, so we can calculate the downward force they exert
on the surrounding air). From the definition of the mixing ratio we know that:

\begin{equation}
  \label{eq:mixing}
  r_v = \frac{\rho_v}{\rho_d} = \frac{\frac{e}{R_v T}}{\frac{p_d}{R_d T}} = \frac{R_d}{R_v} \frac{e}{p-e}
      = \epsilon \frac{e}{p-e} \approx 0.622 \frac{e}{p-e}
\end{equation}

Inverting (\ref{eq:mixing}) gives:

\begin{equation}
  \label{eq:invert}
  \frac{e}{p} = \frac{r_v}{r_v + \epsilon}
\end{equation}

Putting in the equation of state for dry air and group terms, (\ref{eq:dalton}) becomes:

\begin{equation}
  \label{eq:rho2}
      \rho = \frac{p}{R_d T} \left ( 1 - \frac{e}{p} (1 - \epsilon) \right ) + \rho_l + \rho_r + \rho_i
\end{equation}
and using (\ref{eq:invert}):

\begin{equation}
  \label{eq:rho3}
      \rho = \frac{p}{R_d T} \left ( 1 - \frac{r_v}{r_v + \epsilon} (1 - \epsilon) \right ) + \rho_l + \rho_r + \rho_i
\end{equation}

Now divide both sides of (\ref{eq:rho3}) by $\rho_d = (p-e)/(R_d T)$

\begin{equation}
  \label{eq:rho4}
  \frac{\rho}{\rho_d} = \left ( \frac{p}{R_d T_v} \frac{R_d T}{p -e} \right ) =
\frac{p}{R_d T} \frac{R_d T}{p-e} \left [ 1 -  \frac{r_v}{r_v + \epsilon} (1 - \epsilon) \right ] + r_l + r_r  + r_i
\end{equation}
where we've defined the virtual temperature, $T_v$ as the temperature that produces the correct
density $\rho$ for the mixture given the (incorrect) dry air gas constant $R_d$.

Cleaning this up by moving multiplying by $(p-e)/p$:

\begin{equation}
  \label{eq:rho5}
 \frac{T}{ T_v} = 
 \left [ 1 -  \frac{r_v}{r_v + \epsilon} (1 - \epsilon) \right ] +  \frac{p-e}{p} \left [ r_l + r_r  + r_i \right ] 
\end{equation}

But we know that in the atmosphere, $e/p,\ r_l,\ r_r,\ r_i$ are all small (below 0.02) so 
neglect their products, which leaves:

\begin{equation}
  \label{eq:rho6}
 \frac{T}{ T_v} = 
 \left [ 1 -  \frac{r_v}{r_v + \epsilon} (1 - \epsilon) \right ] +   r_l + r_r  + r_i 
\end{equation}
 
Now rearrange (\ref{eq:rho6}), dropping all second order terms:

\begin{equation}
  \label{eq:rho7}
 \frac{T}{T_v} = \frac{r_v + \epsilon - r_v + r_v \epsilon + \epsilon (r_l + r_r + r_i) }{r_v + \epsilon}
\end{equation}

or flipping:

\begin{equation}
  \label{eq:rho8}
 T_v =  T \left ( \frac{r_v + \epsilon}{\epsilon} \right ) \left ( \frac{ 1  }{1 + (r_l + r_r + r_i)}
 \right )
\end{equation}

To get the mixing ratios out of the denominator, use a Taylor series expansion:

\begin{equation}
  \label{eq:taylor}
  f(x)  = f(x_0) + f^\prime(x_0)(x - x_0) + \frac{f^{\prime\prime}(x_0)}{2}(x-x_0)^2 +
            \frac{f^{\prime\prime\prime}(x_0)}{2\cdot3}(x-x_0)^3 + \ldots
\end{equation}

You should show that expanding about $x_0 = 0$ yields:

\begin{equation}
  \label{eq:expan}
\frac{1}{1 + r} = 1 - r + r^2 - r^3 + \ldots  
\end{equation}

So that


\begin{equation}
  \label{eq:rho9}
 T_v \approx  T \left ( \frac{r_v + \epsilon}{\epsilon} \right ) \left ( 1 - (r_l + r_r + r_i)
 \right )
\end{equation}

and rearranging and again dropping second order terms:

\begin{equation}
  \label{eq:rho10}
 T_v \approx   \frac{T (\epsilon + (1-\epsilon) r_v - \epsilon(r_l + r_r + r_i))}{\epsilon}
\end{equation}

and finally:

\begin{equation}
  \label{eq:rho11}
 T_v \approx   T \left ( 1 + \frac{( 1-\epsilon)}{\epsilon} r_v - (r_l + r_r + r_i) \right )
 = T \left ( 1 + 0.608\, r_v - (r_l + r_r + r_i) \right )
\end{equation}

In words \eqref{eq:rho11} says that air becomes more buoyant (effectively ``warmer'')
as $r_v$ increases (because light $H_2O$ replaces heavy $N_2$), and droplets,
drops and ice concentrations decrease (because falling hydrometeors exert downward
drag on the
atmosphere).


\end{document}
%%% Local Variables:
%%% mode: latex
%%% TeX-master: t
%%% End:
