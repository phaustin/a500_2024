\documentclass[12pt]{article}
\usepackage{geometry,fancyhdr,xr,hyperref,ifpdf,amsmath,rcs,shortcuts,amsfonts}
\usepackage{longtable,color,paunits,amsmath,smallsec}
%\usepackage{lastpage,longtable,color,paunits,amsmath,smallsec}
%\usepackage{showlabels}
\geometry{letterpaper,top=50pt,hmargin={20mm,20mm},headheight=15pt} 


\pagestyle{fancy} 

\RCS $Revision: 1.1 $

\RCS $Date: 2007/11/23 17:26:55 $
\fancypagestyle{first}{
\lhead{A500}
\chead{Velocity scales}
%\rhead{page~\thepage/\pageref{LastPage}}
\rhead{page~\thepage}
\lfoot{} 
\cfoot{} 
\rfoot{}
}

\ifpdf
    \usepackage[pdftex]{graphicx} 
    \usepackage{hyperref}
    \pdfcompresslevel=0
    \DeclareGraphicsExtensions{.pdf,.jpg,.mps,.png}
\else
    \usepackage{hyperref}
    \usepackage[dvips]{graphicx}
    \DeclareGraphicsRule{.eps.gz}{eps}{.eps.bb}{`gzip -d #1}
    \DeclareGraphicsExtensions{.eps,.eps.gz}
\fi

\newcommand{\vect}[1]{\mathbf{{#1}}}

\begin{document}

\pagestyle{first}

\begin{center}
Convective and surface velocity scales\\
\end{center}

\section{Convective velocity scale $w_*$}
\label{sec:recap}

Two ways to understand Stull (4.2a) for the convective velocity scale:

\begin{equation}
  \label{eq:wstar}
  w_* = \left (  \frac{ g z_i}{\overline{ \theta_v}} \left (  \overline{ w^\prime \theta_v^\prime } \right )_s \right )^{1/3}
\end{equation}


\begin{enumerate}
\item Intuitively the scales that matter in free convection are the surface buoyancy flux and the inversion height.
From the energy equation 5.1.a we know that the buoyancy flux has the form:

\begin{equation}
  \label{eq:buoyflux}
F_{B} =  \frac{ g }{\overline{ \theta_v}} \left (  \overline{ w^\prime \theta_v^\prime } \right )_s
\end{equation}
with units of $m^2\,s^{-3}$, the only way to get a quantity with units of length/time from $z_i$ and $F_b$ is to take the cube
root of $z_i F_B$.


\item Vertical integration of the TKE equation:  We know that by definition in
a well-mixed layer with temeprature changing uniformly with height the
buoyancy flux has to depend linearly on z (see, e.g. Figure 6.7 p. 223).

\begin{equation}
  \label{eq:linear}
 \left (\overline{ w^\prime \theta_v^\prime } \right ) =  \left (\overline{ w^\prime \theta_v^\prime } \right )_s  \left (    1 - \frac{ z}{z_i}  \right )
\end{equation}

What is the contribution of  $\left (\overline{ w^\prime \theta_v^\prime } \right )$ to the layer TKE?

\begin{equation}
  \label{eq:contrib}
  \begin{split}
  \int_{ 0}^{z_i} \frac{ \partial e}{\partial t} \!\,d z =\int_{ 0}^{z_i} \frac{ g }{\overline{\theta_v}} \left (\overline{ w^\prime \theta_v^\prime } \right )_s \left (   1 - z/z_i  \right )\, dz\\
 = \frac{ g }{\overline{\theta_v}}  \left [ \left [ \left (\overline{ w^\prime \theta_v^\prime } \right )_s z \right ]_s^{z_i}  
 - \left [ \frac{\left (\overline{ w^\prime \theta_v^\prime } \right )_s }{z_i} \frac{ z^2}{2} \right ]_0^{z_i}  \right ] = 
\frac{ g }{\overline{\theta_v}} \frac{\left (\overline{ w^\prime \theta_v^\prime } \right )_s  }{2} z_i  
  \end{split}
\end{equation}
If we ignore the time dependence of $z_i$ and write the vertical average TKE as $\hat{e} = \frac{ 1}{z_i} \int_{ 0}^{z_i} e\!\,d z$
the (\ref{eq:contrib}) becomes:

\begin{equation}
  \label{eq:contrib2}
  \frac{ \partial }{ \partial t} \left ( z_i \hat{e}  \right ) = \frac{ g }{\overline{\theta_v}} 
\frac{  \left (\overline{ w^\prime \theta_v^\prime } \right )_s }{2} z_i
\end{equation}
and since we've assume $z_i$ constant:

\begin{equation}
  \label{eq:contrib3}
  \frac{ \partial  \hat{e}}{ \partial t}   = \frac{ g }{\overline{\theta_v}} 
\frac{  \left (\overline{ w^\prime \theta_v^\prime } \right )_s }{2}
\end{equation}

So we recover the definition of \eqref{eq:final} $w^3_* = \frac{ g }{\overline{\theta_v}} \left (\overline{ w^\prime \theta_v^\prime } \right )_s z_i$ 
if we:

\begin{enumerate}
\item Assume that the layer-averaged TKE scales as a velocity-squared:  $\hat{e} \sim \frac{w_* w_* }{2}$
\item Assume that the principal time scale of interest is the time it takes air moving at velocity $w_*$ to move the
depth of the boundary layer $z_i$:  $t_* = z_i/w_*$.
\end{enumerate}

The we can form new nondimensional variables $\frac{w_* w*} {2} \widetilde{e}=  \hat{e} $ and
$\frac{z_i }{w_*} \widetilde{t} = t$, so that (\ref{eq:contrib3}) gets written as:

\begin{equation}
  \label{eq:contrib4}
\frac{ w_*^2}{2} \frac{w_* }{z_i}  \frac{ \partial }{ \partial \widetilde{t}} \left (\widetilde{e}  \right ) = 
\frac{ g }{\overline{\theta_v}} 
\frac{  \left (\overline{ w^\prime \theta_v^\prime } \right )_s }{2} 
\end{equation}
but if we've done the scaling right, then $\frac{\partial \widetilde{e} }{ \partial \widetilde{t}} \sim 1$, so rearrange
and get:


\begin{equation}
  \label{eq:final}
    w_*^3 = \left (  \frac{ g z_i}{\overline{ \theta_v}} \left (  \overline{ w^\prime \theta_v^\prime } \right )_s \right )
\end{equation}


\end{enumerate}

\section{Convective velocity scale $w_*$ including entrainment}


In reality, in clear air entrainment changes the vertical profile of the buoyancy flux from

\begin{equation}
  \label{eq:linear2}
 \left (\overline{ w^\prime \theta_v^\prime } \right ) =  \left (\overline{ w^\prime \theta_v^\prime } \right )_s  \left (    1 - \frac{ z}{z_i}  \right )
\end{equation}


to

\begin{equation}
  \label{eq:linearent}
 \left (\overline{ w^\prime \theta_v^\prime } \right ) =  \left (\overline{ w^\prime \theta_v^\prime } \right )_s  \left (    1 - \frac{ z}{z_i}  \right )
  + \left (\overline{ w^\prime \theta_v^\prime } \right )_i  
\end{equation}
where $\left (\overline{ w^\prime \theta_v^\prime } \right )_i$ is the buoyancy flux at the inversion, which the dry boundary layer
we will take to be some fraction (and opposite sign) of the  surface buoyancy flux:

\begin{equation}
  \label{eq:propto}
  (\overline{ w^\prime \theta_v^\prime })_i = -A (\overline{ w^\prime \theta_v^\prime })_s
\end{equation}
where A=0.2 for the dry boundary layer

Using \eqref{eq:propto} gives:
\begin{equation}
  \label{eq:linear2}
 \left (\overline{ w^\prime \theta_v^\prime } \right ) =  \left (\overline{ w^\prime \theta_v^\prime } \right )_s  \left (    1 - \frac{12}{10}\frac{ z}{z_i}  \right )
\end{equation}

Taking the vertical integral
\begin{equation}
  \label{eq:linear3}
\int_0^{z_i} \left (\overline{ w^\prime \theta_v^\prime } \right ) dz =  
\int_0^{z_i} \left (\overline{ w^\prime \theta_v^\prime } \right )_s  \left (    1 - \frac{12}{10}\frac{ z}{z_i}  \right )dz
\end{equation}

and if we define the convective velocity scale as:

\begin{equation}
  \label{eq:convent}
  w_*^3 =  2.5 \frac{g}{\theta_0} \int_0^{z_i} \left (\overline{ w^\prime \theta_v^\prime } \right ) dz =  
2.5 \frac{g}{\theta_0} \int_0^{z_i} \left (\overline{ w^\prime \theta_v^\prime } \right )_s  \left (    1 - \frac{12}{10}\frac{ z}{z_i}  \right )dz
= \frac{g}{\theta_0} z_i \left ( \overline{ w^\prime \theta_v^\prime } \right )_s
\end{equation}
we recover the original definition of \eqref{eq:final}.


\section{Surface (friction) velocity scale $u_*$}
\label{sec:surf-frict-veloc}

From p.~67 we know that the friction velociy is a measure of the loss of momentum to the
ground:

\begin{equation}
  \label{eq:friction}
  u_*^2 \sim \langle u^\prime w^\prime \rangle_s
\end{equation}

There is an addition connection between $u_*$ and the lower part of the boundary layer, discussed
in chapter 9 (p.~378).  In statically neutral conditions it is not unusual to find
$\left (\overline{ w^\prime u^\prime } \right )$ constant to witin about 10\% through the
bottom couple of hundred meters (the ``constant flux surface layer'').  Given that, and introducing
the idea of a ``mixing length'' $l$ that represents the size of turbulent eddies that proudce
downgradient turbulent momentum transport towards the surface such that

\begin{equation}
  \label{eq:uprime}
  u^\prime  \sim - l \frac{\partial  \overline{ u}}{\partial z} 
\end{equation}
(see Stull Figure 6.1)

Then the momentum flux is given by 

\begin{equation}
  \label{eq:momflux}
  \overline{ \rho} \left (\overline{ w^\prime u^\prime } \right ) = - \overline{ \rho}
\left (   \overline{ w^\prime l }  \right ) \frac{\partial  \overline{ u}}{\partial z} 
\end{equation}
If the turbulence is fully developed then the eddies are roughly isotropic and we have

\begin{equation}
  \label{eq:wprime}
  w^\prime  \sim - l \frac{\partial  \overline{ u}}{\partial z} 
\end{equation}
so combining (\ref{eq:uprime}) and (\ref{eq:uprime}) we get:

\begin{equation}
  \label{eq:prandtl}
  \left (\overline{ w^\prime u^\prime } \right ) \sim l^2 \left ( \frac{\partial  \overline{ u}}{\partial z}   \right )^2
\end{equation}




 
\subsection{The von Karman constant and the log-layer profile (Stull Section 5.8}
\label{sec:von-karman-constant}

Prandtl assumed that most of the transport at height z was due to eddies
with size $\sim$ z, i.e.~l $sim$ kz where $k$, the von Karman constant is
taken from measurements to be in the range 0.3-0.4.  If we are in the constant
stress layer where $u_*$ doesn't change with height, then

 \begin{equation}
  \label{eq:stress}
  u_* \sim l \left ( \frac{\partial  \overline{ u}}{\partial z} \right )  = kz \frac{\partial  \overline{ u}}{\partial z}
\end{equation}
can easily be integrated from height $z_0$ (the roughness length) to $z$:

\begin{equation}
  \label{eq:integ}
  \int_{ z_0}^{z}\!\, d \overline{u } = \int_{ z_0}^{z}\!\,\frac{u_* }{k}  d \ln z
\end{equation}
which yields:

\begin{equation}
  \label{eq:integ2}
  \overline{u }(z) = \frac{ u_*}{k}  \ln \frac{z }{z_0} 
\end{equation}c


\subsection{Neutral drag coefficient}
\label{sec:neutr-drag-coeff}

The momentum transport in the neutral surface layer can also be written in the form of a \textit{neutral drag coefficient}
$C_{DN}$ (Stull 7.4.1a):

\begin{equation}
  \label{eq:drag}
  u_*^2 = - C_{DN} \overline{ M}^2
\end{equation}
where $M^2 = u^2 + v^2$, or using (\ref{eq:integ2}) we can get Stull 7.4.1i:

\begin{equation}
  \label{eq:cdn}
  C_{DN} = k^2 \left ( \ln \left ( \frac{ z}{z_0}   \right )  \right )^{-2}
\end{equation}
which is used in the drag laws (p. 262):


\begin{gather}
  \overline{u^\prime w^\prime}_s = -C_D \overline{M} \overline{U}\\
\overline{v^\prime w^\prime}_s = -C_D \overline{M} \overline{V}
\end{gather}

Next lecture we will show that 
$z_i$, $w_*$  and $u_*$ are related by the Obukhov Length $L$,
(Stull Section 5.7)

\begin{equation}
  \label{eq:mo}
  \frac{ z_i}{L} = -k \frac{ w^3_*}{u_*^3} 
\end{equation}


\end{document}



%%% Local Variables:
%%% mode: latex
%%% TeX-master: t
%%% End:

