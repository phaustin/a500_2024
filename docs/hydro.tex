\documentclass[12pt]{article}
\usepackage{geometry,fancyhdr,xr,hyperref,ifpdf,amsmath,rcs,indentfirst}
\usepackage{lastpage,longtable,Ventry,url,paunits,shortcuts,smallsec,color,tightlist,float}
\geometry{letterpaper,top=50pt,hmargin={20mm,20mm},headheight=15pt} 
\usepackage[stable]{footmisc}
\pagestyle{fancy} 

\RCS $Revision: 1.5 $
\RCS $Date: 2002/01/09 03:50:54 $

\fancypagestyle{first}{
\lhead{Hydrostatic balance}
\chead{}
\rhead{page~\thepage/\pageref{LastPage}}
\lfoot{} 
\cfoot{} 
\rfoot{}
}

\ifpdf
    \usepackage[pdftex]{graphicx} 
    \usepackage{hyperref}
    \pdfcompresslevel=0
    \DeclareGraphicsExtensions{.pdf,.jpg,.mps,.png}
\else
    \usepackage{hyperref}
    \usepackage[dvips]{graphicx}
    \DeclareGraphicsRule{.eps.gz}{eps}{.eps.bb}{`gzip -d #1}
    \DeclareGraphicsExtensions{.eps,.eps.gz}
\fi


\begin{document}
\newcommand{\vect}[1]{\boldsymbol{\vec{#1}}}
\pagestyle{first}

\section{Hydrostatic balance}


For the boundary layer, it is a very good approximation to
assume \textit{hydrostatic balance}, which is the statement that an
the vertical pressure differential across a layer provides exactly the
amount of force necessary to balance gravity:

  \begin{figure}[H]
    \begin{center}
       \input hydro.pstex_t
      \caption{hydrostatic balance for a $1 \times dz\  \un{m^3}$ layer}
      \label{fig:hydro}
    \end{center}
  \end{figure}

\vspace{0.1in}

In symbols, the balance shown by Figure \ref{fig:hydro} implies that:

\begin{equation}
  \label{eq:hydro}
  \frac{dp}{dt} = - \rho g \frac{dz}{dt}
\end{equation}
\textit{Question:  is this the same pressure p as the local pressure given by the equation of state: $p=\rho R_d T$?}

\section{Scale Heights}
\label{sec:scale-heights}

In Chapter 3 (p.~81) Roland introduces the idea of a characteristic
length scale for the variation of density in the atmosphere: $\frac{d \ln \rho }{dz}$ .  Here
is how that arises from the hydrostatic approximation:

\begin{enumerate}
\item Do pressure first: Rewrite (\ref{eq:hydro}) using the ideal gas law:
  \begin{align}
  dp =& - \rho g dz = - \frac{p}{R_d T}  g dz\\
  d\ln p =& - \frac{g }{R_d T} dz\\
  \int_{p_0}^{p}\!\,d \ln p =& - \int_{0 }^{z}\!\frac{g }{R_d T} \,dz^\prime = - \int_{0 }^{z}\!\frac{1}{H} dz^\prime \label{eq:hydro1}
  \end{align}
where $H=R_d T/g$ is the pressure scale height.  Since temperature (and gravity) are changing with height,
we need to work with a vertical average:

\begin{align}
\frac{ 1}{\overline{H}} =&  \overline{ \left ( \frac{1 }{H} \right )} = \frac{\int_{0 }^{z}\!\frac{1}{H} dz^\prime  }{z-0} 
\end{align}
Which allows us to easily integrate (\ref{eq:hydro1}):

\begin{align}
  \ln p/p_0 =& - \frac{z }{\overline{H }} \\
  p =& p_0 \exp \left ( - \frac{z }{\overline H} \right )
\end{align}


\item Now repeat this for density:

  \begin{align}
 \frac{dp }{dz}  =& \frac{d }{dz}  (\rho R_d T) = R_d \left ( \frac{d\rho }{dz} T 
              + \rho \frac{ dT}{dz} \right )  = - \rho g \\
\frac{d\rho }{dz}  =& -\frac{\rho }{T}  \left ( \frac{g }{R_d} + \frac{ dT}{dz} \right )\\
\frac{d\rho }{\rho} =& - \left ( \frac{1 }{H} + 
                   \frac{1 }{T} \frac{dT }{dz} \right ) dz \equiv - \frac{dz }{H_\rho} 
  \end{align}
The vertical average is taken in exactly the same way as before to get the density profile:

\begin{align}
  \ln \rho/\rho_0 =& - \frac{z }{\overline{H_\rho }} \\
  \rho =& \rho_0 \exp \left ( - \frac{z }{\overline H_\rho} \right )
\end{align}

Running the python code for the midlatitude summer sounding gives
$H\approx7.8$ km and $H_\rho \approx 9.5$ km.


\end{enumerate}



\end{document}

%%% Local Variables:
%%% mode: latex
%%% TeX-master: t
%%% End:
