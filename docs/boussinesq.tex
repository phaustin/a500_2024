\documentclass[12pt]{article}
\usepackage{geometry,fancyhdr,xr,hyperref,ifpdf,amsmath,rcs,indentfirst}
\usepackage{lastpage,longtable,Ventry,url,paunits,shortcuts,smallsec,color,tightlist,float}
\geometry{letterpaper,top=50pt,hmargin={20mm,20mm},headheight=15pt} 
\usepackage[stable]{footmisc}

\pagestyle{fancy} 

\RCS $Revision: 1.5 $
\RCS $Date: 2002/01/09 03:50:54 $

\fancypagestyle{first}{
\lhead{Anelastic and Bousinesq approximation}
\chead{}
\rhead{page~\thepage/\pageref{LastPage}}
\lfoot{} 
\cfoot{} 
\rfoot{}
}

\ifpdf
    \usepackage[pdftex]{graphicx} 
    \usepackage{hyperref}
    \pdfcompresslevel=0
    \DeclareGraphicsExtensions{.pdf,.jpg,.mps,.png}
\else
    \usepackage{hyperref}
    \usepackage[dvips]{graphicx}
    \DeclareGraphicsRule{.eps.gz}{eps}{.eps.bb}{`gzip -d #1}
    \DeclareGraphicsExtensions{.eps,.eps.gz}
\fi


\begin{document}
\newcommand{\vect}[1]{\boldsymbol{\vec{#1}}}
\pagestyle{first}


\section{Scaling}
\label{sec:thermodynamics}


Here's my version of  
\href{https://hogback.atmos.colostate.edu/group/dave/QuickStudies.html}{David Randall's}
discussion of the anelastic and Boussinesq approximations.  

It differs from Stull p. 84 in that rather than starting with a Reynold's decomposition it takes
the more general approach of starting with perturbations around a hydrostatic
base state.  This has the advantage of not requiring the assumption that the
atmosphere is in hydrostatic equilibrium.


\subsection{Conservation equations}
\label{sec:cons-equat}


\begin{align}
  \text{entropy:\ }& \frac{ ds}{dt}=\frac{ q}{T} + \nu_s \nabla^2 s\\
  \text{water vapor:\ }& \frac{ dq_v}{dt}= \nu_s \nabla^2 q_v \\
  \text{momentum:\ }& \rho \frac{ d \vect{v}}{dt}= -\nabla p + \rho \vect{g} + 
             \mu \nabla^2 \vect{v} + \rho f_c \vect{k} \times \vect{v} \\
 \intertext{and the \textit{flux form} of the continuity equation}
\text{mass:\ }&= \frac{\partial \rho }{\partial t} + \nabla \rho \cdot \vect{v} = 0
\end{align}

\subsection{Dropping the viscous terms: Reynolds number: Stull p.~93}
\label{sec:stull93}

What about the viscosity?  $\nu_{air}=\mu/\rho \sim 1.5 \times 10^{-5}\ \ms$.  
For our problems typically $| \vect{v} | \sim 5\ \ms$, $L \sim 100\ $ m, 
so scale the viscous term in comparison to $\rho \frac{ du}{dt}$, as
represented by the advection term: $\rho \vect{v} \cdot \nabla \vect{v}$. 


First, consider viscous dissipation:
\begin{gather}
  \frac{\mu }{\rho} \nabla^2 \vect{v} \sim \frac{10^{-5} }{(100)^2} \cdot 5 \sim 10^{-8}\ \un{m\,s^{-2}}
\end{gather}
and compare this to the acceleration scale (assuming local steady state where $ \partial u/\partial t \sim 0$ )

\begin{gather}
  \frac{ d \vect{v}}{dt} \sim |u|/(L/|u|) \sim 5/(100|5) \sim 0.25
\end{gather}

The ratio of these two numbers is 
the Reynold's number = advection/diffusion =

\begin{gather}
Re=(\rho \vect{v} \cdot \nabla \vect{v} /\mu \nabla^2 \vect{v}) \sim  |u| L/\nu \sim 10^{8}
\end{gather}
So neglect molecular viscosity/diffusivity compared to other momentum sources
and sinks

\subsection{Hydrostatic balance and the equations of state}
\label{sec:energy}

On p.~81 Stull expands the density, virtual temperature and pressure into
``mean and turbulent parts''.  The people who specialize in deriving
these equations take a slightly different approach, which we'll follow
here:

\begin{itemize}
\item Expand the thermodynamic variables about a hydrostatic base state at rest
\item Expand the deviation from the hydrostatic base state into mean and turbulent parts
\end{itemize}
What is the advantage of this approach?  Basically it produces equations that are
easier to extend to non-hydrostatic atmospheres.

So write each of the thermodynamic variables as a combination of a static reference profile
and a departure from the reference.  This is \textit{not} a Reynold's expansion since
the base state isn't an average, it's  a Taylor's series approximation in
two dimensions (see the 
\href{https://www.dropbox.com/scl/fi/zwjwuzed05cvya6xydj9e/taylor_series.pdf?rlkey=jt7zu8gedkwxaosty87e7h0o8&dl=0}{Taylor series notes}  from week 4)


Look at
the total entropy $s_T$, pressure $p_T$, temperature $T_T$, density $\rho_T$ and specific
volume $\alpha_T = 1/\rho_T$:

\begin{subequations}
\begin{align}
  s_T =& s_0(z) + s(x,y,z,t) = c_p \ln \theta_0 (z) + c_p \ln \theta (x,y,z,t)\\ 
  p_T =& p_0(z) + p(x,y,z,t) \\
  T_T =& T_0(z) + T(x,y,z,t) \\
  \rho_T =& \rho_0(z) + \rho(x,y,z,t) \\
  \alpha_T =& \alpha_0(z) + \alpha(x,y,z,t)
\end{align}
\end{subequations}

Require that the total atmosphere and the base state both satisfy the
equation of state:

\begin{subequations}

\begin{align}
  p_T =& \rho_T R_d T_T\\
  p_0 +  p =& R_d(\rho_0 + \rho)(T_0 + T)\\
  p_0 +  p =& R_d(\rho_0 T_0 + \rho_0 T + \rho T_0 + \rho T)\\
  1 + \frac{p}{p_0(z)}  =& R_d \left ( \frac{\rho_0 T_0}{p_0}  + \frac{\rho_0 T}{p_0} + \frac{\rho T_0}{p_0}
           + \frac{\rho T}{p_0} \right )\\
\intertext{and since $p_0 = R_d \rho_0 T_0$}
1 + \frac{p}{p_0(z)}  =& \left ( 1  + \frac{ T}{T_0} + \frac{\rho }{\rho_0} +  \frac{\rho T}{\rho_0 T_0} \right )
\end{align}
\end{subequations}

Recognize that the atmosphere is close to hydrostatic balance, so that $\frac{\rho}{\rho_0}$ and
$\frac{T}{T_0}$ are both small.  Drop the last term and get the perturbation equation of state:


\begin{align}
  \frac{p}{p_0(z)} \approx & \frac{T}{T_0(z)}  + \frac{\rho}{\rho_0(z)} 
\end{align}
Note that this is not what Roland has on p.~81, (3.3.1a) since we haven't done any Reynold's averaging
yet.

\section{Mass conservation: Bousinesq and anelastic versions}
\label{sec:cont}

  Separate  out the vertical and horizontal terms in the continuity equation for the total density:

\begin{equation}
\frac{\partial \rho_T}{\partial t} + \nabla_H \cdot  ( \rho_T \vect{v}_H) + 
          \frac{\partial (\rho_T w)}{\partial z} = 0
\end{equation}

\begin{equation}
  \label{eq:split}
\frac{(\partial \rho_0 + \rho)}{\partial t} + 
       \nabla_H \cdot  (( \rho_0 + \rho) \vect{v}_H) + 
          \frac{\partial ((\rho_0 + \rho) w)}{\partial z} = 0
  \end{equation}
(note that our base state is at rest, so $w_0=0$)

Now expand, using the fact that the reference density $\rho_0$ depends only
on z to throw out some derivatives:
\begin{equation}
\label{eq:bigone}
\underbrace{\frac{\partial \rho }{\partial t}}_1 + 
\underbrace{\rho_0(z) \nabla_H \cdot \vect{v}_H}_2 + 
\underbrace{\rho \nabla_H \cdot \vect{v}_H}_3 +
\underbrace{\vect{v}_H \cdot \nabla \rho}_4 +
\underbrace{w \frac{\partial \rho_0(z) }{\partial z}}_5 +
\underbrace{\rho_0(z) \frac{\partial w }{\partial z}}_6 +
\underbrace{w \frac{\partial \rho }{\partial z}  }_7 +
\underbrace{\rho \frac{\partial w }{\partial z}}_8 = 0
 \end{equation}

So how many of these 8 terms to we keep?  
We will work through 2 different approximations:  the Bousinesq approximation,
in which we drop everything except terms (2) and (6), and the anelastic
approximation, in which we also keep (5).
How do we justify eliminating the other 5 terms?  We will compare them
to (6), which controls the vertical divergence/convergence of mass.

\subsection{Detour: Buoyancy}
\label{sec:buoyancy}

Before we go any further, we need a specific definition of the
buoyancy.  Start with Newton's second law:

\begin{equation}
  \label{eq:newton}
  m a = F
\end{equation}

Rewrite this in for  the vertical velocity component:
and gravity:
\begin{equation}
  \label{eq:newt2}
  \rho_T \frac{ dw}{dt}  = - \frac{ \partial p_T}{\partial z} - \rho_T g
\end{equation}
where as usual $p_T = p_0 + p$ and $p_0$ is the pressure in hydrostatic
balance with the density $\rho_0(z)$.

Hydrostatic balance means that:

\begin{gather}
  dp_0 = - \rho_0 g dz\\
  -\frac{ 1}{\rho_0} \frac{ \partial p_0}{\partial z} -g = 0 \label{eq:one}
\end{gather}

So expand (\ref{eq:newt2}):

\begin{equation}
  \label{eq:newt3}
   \frac{ dw}{dt}  = - \frac{1 }{(\rho_0 + \rho)} 
\frac{ \partial }{\partial z} (p_0 + p ) -  g
\end{equation}
and expand the density term in a Taylor's series:

\begin{equation}
  \label{eq:taylor}
  \frac{ 1}{(\rho_0 + \rho)} = \frac{ 1}{\rho_0} 
\left ( \frac{ 1}{1 + \rho/\rho_0}   \right ) = 
\frac{ 1}{\rho_0} \left [ 1 - \frac{ \rho}{\rho_0} + 
\left ( \frac{ \rho}{\rho_0} \right )^2 + \ldots \right ]
\end{equation}


If we keep only the first order terms in (\ref{eq:taylor}) and plug into
(\ref{eq:newt3}), again dropping second order terms like $wp$,
we get
 

\begin{equation}
  \label{eq:newt4}
   \frac{ dw}{dt}  = \underbrace{- \frac{1 }{\rho_0} 
\frac{ \partial p_o }{\partial z} -  g}_a
- \frac{1 }{\rho_0} \frac{ \partial p }{\partial z} +
 \underbrace{\frac{1 }{\rho_0} \frac{ \partial p_0 }{\partial z}}_b \left ( \frac{ \rho}{\rho_0} \right )
\end{equation}
Now I can use hydrostatic balance twice: once to eliminate (a), and again
to write (b) as $-g$.  Doing this I get:


\begin{equation}
  \label{eq:newt5}
   \frac{ dw}{dt}  = - \frac{1 }{\rho_0} \frac{ \partial p }{\partial z} 
- g \left ( \frac{ \rho}{\rho_0} \right ) =
- \frac{1 }{\rho_0} \frac{ \partial p }{\partial z} 
+ B
\end{equation}
where the buoyancy, $B$, is defined by:

\begin{equation}
  \label{eq:B}
B = - g \left ( \frac{ \rho}{\rho_0} \right )
\end{equation}


\subsection{Scaling equation (\ref{eq:bigone})}


To scale the 8 terms, we need to:

\begin{enumerate}
\item write derivatives as finite differences, i.e.~ $\frac{\partial w }{\partial z}  \sim \frac{ \Delta w}{\Delta z}$
\item Make guess about how big terms like $\Delta w$ and $\Delta z$ could
get for the phenomena we are interested in
\end{enumerate}


\subsubsection{Term 1:  $\frac{\partial \rho }{    \partial t}$}
\label{sec:term-1}


We want to guestimate 1) using dimensional values:

$\frac{\partial \rho }{\partial t} = \frac{\Delta \rho}{\Delta \tau}$:

\begin{enumerate}
\item Let $\Delta \rho \ll \rho_0$ (assume small depatures from hydrostatic balance)

\item $\Delta \rho \sim |\rho |$: density changes as big as the difference
between the total density $\rho_T$ and the base state density $\rho_0$
\item Compare term (1) to term (6): 
$\rho_0(z) \frac{\partial w }{\partial z} \sim \rho_0 \frac{\Delta w }{\Delta z}
\sim \rho_0 \frac{ |w|}{D}$  
where $|w|$, $D$ and $\tau$ are typical velocity, length and time scales
(whatever we can defend).  For example, assume that we are interested
in a boundary layer dominated by eddies of size $D \sim 100$ m,
which turnover at a velocity of about 0.1 \ms, yielding a timescale
of $\tau \sim 100$ s.  

The eddy turnover velocity is going to be slower than the vertical
velocity of a typical updraft, which is driven by buoyancy.  We expect
these buoyant updrafts to scale as:

\begin{equation}
  \label{eq:buoy}
  \frac{\partial w }{\partial t} \sim \frac{|w| }{\tau} \sim B \sim -\frac{ g \rho}{\rho_0(z)} 
\end{equation}
where we've neglected the vertical pressure gradient in (\ref{eq:newt5})

So compare $\underbrace{\frac{\partial \rho }{\partial t}}_1$ 
against the yardstick:

\begin{equation}
  \label{eq:yardstick}
  \frac{ \frac{ \partial \rho}{\partial t} }{\rho_0 \frac{ \partial w}{\partial z} } \sim 
\frac{ |\rho|/\tau}{|w|\rho_0/D} \sim
\frac{ |\rho|}{\rho_0} \frac{ D}{|w| \tau}  
\end{equation}
but $|w| \sim \tau g \frac{ |\rho|}{\rho_0}$ so:

\begin{equation}
  \label{eq:term1comp}
  \frac{ \frac{ \partial \rho}{\partial t} }{\rho_0 \frac{ \partial w}{\partial z} } \sim 
\frac{|\rho| }{\rho_0} \frac{D }{\tau^2} \frac{ 1}{g \frac{ |\rho| }{\rho_0} } 
= \frac{D }{g \tau^2} 
\end{equation}
so we can ignore term 1 if $D/(g \tau^2)$  is $\ll 1$.  This is Stull point 2) on p.~81.

\end{enumerate}

In other words, we require that the timescale we're interested is
``much larger'' than $\sqrt{10} \sim 3 s$, since that means

\begin{equation}
  \label{eq:comp1b}
  \tau^2 \gg \frac{ D}{g}  \sim \frac{100 }{10} \sim 10\ s^2
\end{equation}
and as we increase the size of the scale of interest $D$, we have
to recognize that we can only ask questions about longer timescales
(several minutes).  It is possible to show explicitly that
this means that both the anelastic and Boussinesq approximations
filter out sound waves (see Randall).


\subsubsection{The horizontal terms}
\label{sec:horizontal-terms}


Next look at the horizontal terms in (\ref{eq:bigone}):

\begin{equation}
  \label{eq:horiz}
  \underbrace{\rho_0(z) \nabla_H \cdot \vect{v}_H}_2 + 
\underbrace{\rho \nabla_H \cdot \vect{v}_H}_3 +
\underbrace{\vect{v}_H \cdot \nabla \rho}_4 
\end{equation}
where the $H$ means ``horizontal''
Keep term (2) for now since it's multiplied by $\rho_0$ and we're
assuming $\rho_0 \gg \rho$

For terms 3) and 4) assume $\vect{v}_H \sim 5\ \ms = |v|$ and $L$ =
the horizontal scale for synoptic disturbances $\sim$ 100 km (rainbands,
etc., so that  $\nabla_H \sim 1/L$l.

So therm 3) scale as 

\begin{equation}
  \label{eq:scaleH}
  \frac{\rho \nabla_H \cdot \vect{v}_H }{\rho_0 \frac{ \partial w}{\partial z} }
\sim \frac{|\rho| }{\rho_0} \frac{|v| }{L}  \frac{ D}{w} 
\sim \frac{|\rho| }{\rho_0} \frac{|v| }{L} \frac{D }{\tau g \frac{|\rho| }{\rho_0} }  \sim \frac{|v| D }{L \tau g} 
\end{equation}
If $\tau \sim 100$ s this is $\frac{5 }{100 \times 10^3} 
\frac{100 }{100 \cdot 10}  \sim 5 \times 10^{-4}$, so we can safely drop
3) and,
 if you repeat this process, 4).

 What about $\underbrace{w \frac{\partial \rho_0(z) }{\partial z}}_5$?  As we saw in scale height
notebook the density for the
hydrostatic base state is well-approximated by:

\begin{equation}
  \label{eq:hydrodens}
  \rho_0 (z) = \rho_{0s} \exp \left ( \frac{-z }{H_\rho} \right )
\end{equation}
where $H_\rho$, the density scale height $\sim$ 8 km.  So 
$\frac{\partial \rho_0 }{\partial z}  \sim \frac{\rho_{0s} }{H_\rho}$
and

\begin{equation}
  \label{eq:term5}
 \frac{w \frac{\partial \rho_0(z) }{\partial z} }{\text{term 6}}   \sim \frac{|w| 
\frac{\rho_{0s} }{H_\rho}  }{\rho_0 \frac{ |w|}{D}  } \sim \frac{ D}{H_\rho} 
\end{equation}
this term is small asl on as the largest eddy size is a few hunder meters.
This is the ``shallow motion'' approximation on p.~81.
We cannot neglect this term for convective clouds, which might have thicknesses
of several km or more.  


\subsubsection{Terms 7 and 8}
\label{sec:terms-7-8}


What about the final two terms:  $\underbrace{w \frac{\partial \rho }{\partial z}  }_7 +
\underbrace{\rho \frac{\partial w }{\partial z}}_8 $?  For problems the atmosphere is
close to hydrostatic, so $\rho/\rho_0$ is small.  If that's the case,
we can ignore 7) because:

\begin{equation}
  \label{eq:term7}
  \frac{ w \frac{\partial \rho }{\partial z} }{\rho_0 \frac{ \partial w}{\partial z}} \sim \frac{ |w|\frac{ |\rho|}{D} }{\rho_0 \frac{|w| }{D} }  
\sim \frac{ |\rho|}{\rho_0}  \ll 1
\end{equation}
and you should show that the same reasoning holds for 8).


That leaves us with two different versions of the continuity equation:

\subsubsection{Anelastic}
\label{sec:anelastic}

\begin{equation}
  \label{eq:anel1}
  \underbrace{\rho_0(z) \nabla_H \cdot \vect{v}_H}_2 + 
\underbrace{w \frac{\partial \rho_0(z) }{\partial z}}_5 +
\underbrace{\rho_0(z) \frac{\partial w }{\partial z}}_6 =0
\end{equation}
Since $\vect{v}_H \cdot \nabla_H \rho_0$=0, we can add this
term into (\ref{eq:anel1}) and use the chain rule to wind up
with:

\begin{equation}
  \label{eq:anel2}
  \nabla ( \rho_0 \vect{v} ) = 0
\end{equation}

\subsubsection{Boussinesq}
\label{sec:boussinesq}

If we are in a situation where $\frac{D }{H_\rho} \ll 1$ as well,
then we can drop 5) and we're left with:

\begin{equation}
  \label{eq:bous1}
  \underbrace{\rho_0(z) \nabla_H \cdot \vect{v}_H}_2 + 
\underbrace{\rho_0(z) \frac{\partial w }{\partial z}}_6 =0
\end{equation}
and putting the gradient back together gives:

\begin{gather}
  \label{eq:bous2}
\rho_0(z) \nabla \vect{v} =0\\
\intertext{or dividing by $\rho_0$: }
\nabla \vect{v} =0
\end{gather}
So the Bousinesq approximation yields the incompressible continuity equation.


\subsection{Back to Buoyancy}
\label{sec:back-buoyancy}

On p.~82-83 Stull makes an argument for neglecting pressure pertubations compared 
to the perturbations in density and temperature for Reynold's averaged
quantities.  But remember we still haven't done the Reynolds averaging 
in this derivation.  Instead of following him and neglecting pressure perturbations
(3.3.1.d and 3.3.1.e),  we can develop a more accurate form of the 
buoyancy with a little algebra.

Recall~(\ref{eq:newt5})
\begin{equation}
  \label{eq:buoyb}
   \frac{ dw}{dt}  = - \frac{1 }{\rho_0} \frac{ \partial p }{\partial z} 
- g \left ( \frac{ \rho}{\rho_0} \right ) =
- \frac{1 }{\rho_0} \frac{ \partial p }{\partial z} 
+ B
\end{equation}

Use the chain rule to show that (\ref{eq:buoyb}) is identical to the
following:

\begin{equation}
  \label{eq:buoyc}
   - \frac{1 }{\rho_0} \frac{ \partial p }{\partial z} 
+ B =
- \frac{\partial  }{\partial z} \left ( \frac{  p }{\rho_0} \right )
- \frac{p }{\rho_0} \frac{\partial  }{\partial z} (\ln \rho_0)
+ B
\end{equation}
Why is this an improvement?  Recall the definition of the potential
temperature from 
\href{https://www.dropbox.com/scl/fi/zuk9evzf47qdsxi9tvgx7/thermo.pdf?rlkey=hbz3bpt6gxv5ly8rg1njfj9e4&dl=0}{the thermo notes}:

\begin{equation}
  \label{eq:theta}
  \theta_T \equiv T_T \left ( \frac{p_{00} }{p_T}  \right )^{R_d/c_p} = 
T_T \left ( \frac{p_{00} }{p_T}  \right )^\kappa
\end{equation}


In the problem set you're asked to show using the ideal gas law
that:

\begin{equation}
  \label{eq:ideal}
  c_p \ln \theta_T = c_v \ln p_T - c_p \ln \rho_T - c_p \ln R_d 
- R_d \ln p_{00}
\end{equation}
and for the base state:

\begin{equation}
  \label{eq:ideal2}
  c_p \ln \theta_0 = c_v \ln p_0 - c_p \ln \rho_0 - c_p \ln R_d 
- R_d \ln p_{00}
\end{equation}
and that:

\begin{equation}
  \label{eq:perturbTheta}
  c_p \frac{ \theta}{\theta_0} \approx c_v \frac{ p}{p_0}  - 
c_p \frac{ \rho}{\rho_0} 
\end{equation}

%check the derivation
Defining $\gamma=c_p/c_v$ and inserting these into (\ref{eq:buoyc}) along with
the hydrostatic equation yields:

\begin{multline}
  \label{eq:buoyd}
   - \frac{1 }{\rho_0} \frac{ \partial p }{\partial z}  - g \left ( \frac{\rho}{\rho_0} \right )
\approx -\frac{\partial  }{ \partial z} \left ( \frac{ p}{\rho_0} \right )
+ \frac{p }{\rho_0}  \frac{ \partial }{\partial z} 
\left ( \ln \theta_0 - \frac{1}{\gamma} \ln p_0 \right )
+ g \left ( \frac{ \theta}{\theta_0} - \frac{ 1}{\gamma} \frac{p }{p_0} \right ) \\
= \underbrace{-\frac{\partial  }{ \partial z} \left ( \frac{ p}{\rho_0} \right )}_I
+ \underbrace{\frac{p }{\rho_0}  \frac{ \partial }{\partial z} 
\left ( \ln \theta_0  \right )}_{II}
+ \underbrace{g \left ( \frac{ \theta}{\theta_0} \right ) }_{III}
\end{multline}

We can eliminate term II) by recognizing that $\frac{\partial  }{\partial z} 
(\ln \theta_0$ ) = $\frac{1 }{\theta_0}\frac{d \theta_0 }{dz} = \frac{1 }{H_\theta}  $ defines another scale height, like $H_p$ or $H_\rho$. It is typically bigger than 8 km,
so we can use this to compare terms I) and II):


 \begin{equation}
   \label{eq:compare}
\frac{ \frac{p }{\rho_0}  \frac{ \partial }{\partial z} ( \ln \theta_0 )}
{\frac{\partial  }{ \partial z} \left ( \frac{ p}{\rho_0} \right )}
\sim \frac{\frac{p }{\rho_0} \frac{1 }{H_\theta}  }
{\frac{p }{\rho_0} \frac{1 }{D} } = \frac{D}{H_\theta} = \frac{100 }{8000}  
\end{equation}
so it's safe to drop term II) and write the vertical velocity
equation for the anelastic approximation as 


\begin{equation}
  \label{eq:anelvert}
\frac{dw }{dt}  =   - \frac{1 }{\rho_0} \frac{ \partial p }{\partial z}  - g\left ( \frac{ \rho }{\rho_0} \right )
\approx -\frac{\partial  }{ \partial z} \left ( \frac{ p}{\rho_0} \right )
+ g \left ( \frac{ \theta}{\theta_0} \right )
=  -\frac{\partial  }{ \partial z} \left ( \frac{ p}{\rho_0} \right )
+ B_a
\end{equation}
where the anelastic buoyancy term is defined by
$B_a=g \left ( \frac{ \theta}{\theta_0} \right )$

\section{Energy and entropy}
\label{sec:energy-entropy}

Recall from the \href{https://www.dropbox.com/scl/fi/zuk9evzf47qdsxi9tvgx7/thermo.pdf?rlkey=hbz3bpt6gxv5ly8rg1njfj9e4&dl=0}{the thermo notes} that for a reversible process we can write the
second law as:

\begin{equation}
  \label{eq:2nd}
  \frac{d \ln \theta_T }{dt} = \frac{q }{\rho_T c_p T_T} 
\end{equation}
where the subscript $T$ as before indicates the sum of the
hydrostatic base state (i.e.~$\theta_0$  and perubation (i.e.~$\theta$)).

Plugging the $\theta_0 + \theta$, $\rho_0 + \rho$ and $T_0 + T$ 
expansions into (\ref{eq:2nd}) and rearranging gives 
\href{https://www.dropbox.com/scl/fi/ugg14cvujp9m5b99c2vbf/bannon95a.pdf?rlkey=d8v4qvqpncduunwqpb1wdgtz1&dl=0}{Bannon(2.3c)}:

\begin{equation}
  \label{eq:bannon2nd}
  \frac{D \theta }{Dt}  + w \frac{ d \theta_0}{dz}  = \frac{\theta_0 q}{\rho_0 c_p T_0} 
\end{equation}



\end{document}



%%% Local Variables:
%%% mode: latex
%%% TeX-master: t
%%% End:
